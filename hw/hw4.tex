\documentclass[12pt, leqno]{article}
\input{common}

\begin{document}

\hdr{HW 4}

\paragraph*{1: Crossing cubics}
A {\em Bezier curve of degree $n$} is the curve traced out by
\[
  f(t) = \sum_{k=0}^n p_i B_i^n(t), \quad t \in [0,1]
\]
where the points $p_i \in \bbR^2$ are {\em control points} and
the functions $B_i^n(t)$ are the {\em Bernstein polynomials}
\[
  B_i^n(t) = C^n_i (1-t)^{n-i} t^i, \quad C^n_i = \frac{n!}{i!(n-i)!}.
\]
A common type of Bezier curve in computer graphics is the {\em cubic
  Bezier curve} defined by four control points.  Complete the
following function to compute the intersection of two such curves
\begin{lstlisting}
  function [x,y] = bezier_intersect(pf, pg)
  %
  % Attempt to compute the intersection of the cubic Bezier curves
  % defined by the columns of pf and pg (each of dimension 2-by-4).
  % Use Newton iteration starting from the initial guess of t = 0.5
  % for both curves.
\end{lstlisting}
Illustrate with an example that your solution works correctly.

\paragraph*{2: Funky fixed point}
Argue that the iteration
\begin{align*}
  3x_{k+1} + 2y_{k+1} &= \cos(x_k) \\
  2x_{k+1} + 4y_{k+1} &= \cos(y_k)
\end{align*}
converges to a unique fixed point $(x_*, y_*)$, regardless of
the initial point, and that $\|e_{k+1}\|_2 < 0.7 \|e_k\|_2$, where
$e_k = (x_k-x_*, y_k-y_*)$.  Starting from the point $(1,1)$,
draw a semi-logarithmic plot of the error versus $k$ to illustrate
the convergence.

\end{document}
