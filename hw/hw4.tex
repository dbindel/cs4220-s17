\documentclass[12pt, leqno]{article}
\usepackage{amsfonts}
\usepackage{amsmath}
\usepackage{amssymb}
\usepackage{fancyhdr}
\usepackage{hyperref}
\usepackage{tikz}
\usepackage{pgfplots}
\usepackage{listings}

\newcommand{\bbR}{\mathbb{R}}
\newcommand{\bbC}{\mathbb{C}}
\newcommand{\calV}{\mathcal{V}}
\newcommand{\calW}{\mathcal{W}}
\newcommand{\ddiag}{\operatorname{diag}}
\newcommand{\fl}{\operatorname{fl}}
\newcommand{\macheps}{\epsilon_{\mathrm{mach}}}
\newcommand{\matlab}{\textsc{Matlab}}

\newcommand{\hdr}[2]{
  \pagestyle{fancy}
  \lhead{Bindel, Spring 2016}
  \rhead{Numerical Analysis (CS 4220)}
  \fancyfoot{}
  \begin{center}
    {\large{\bf Notes for #1}}
  \end{center}
  \lstset{language=matlab,columns=flexible}  
}


\begin{document} \hdr{HW 3}{Fri, Apr 8}

% One question on 1D root finding
%
\paragraph*{1: Go with the flow}
The Darcy friction coefficient $f$ for turbulent flow in a pipe is
defined in terms of the Colebrook-White equation for large Reynolds
number Re (greater than 4000 or so):
\[
\frac{1}{\sqrt{f}} =
  -2 \log_{10} \left( \frac{\epsilon/D_h}{3.7} +
  \frac{2.51}{\mathrm{Re} \sqrt{f}} \right)
\]
Here $\epsilon$ is the height of the surface roughness and $D_h$ is the
diameter of the pipe. For a 10 cm pipe with 0.1 mm surface roughness,
find $f$ for Reynolds numbers of $10^4$, $10^5$, and $10^6$.
Ideally, you should use a Newton iteration with a good initial guess;
it may help to reformulate the problem in terms of a variable other
than $f$.

% One question on fixed-point iteration
%
\paragraph*{2: Funky fixed point}
Suppose we wish to find the fixed point $Ax = f(x)$ by the iteration
\[
  A x^{k+1} = f(x^k),
\]
where $f$ is Lipschitz (in the two-norm) with
constant $M < \sigma_{\min}(A)$.  Write an error iteration
and analyze it to argue that the fixed point iteration converges.

% One question on multivariate calculus
%
\paragraph*{3: Eine kleine Nacht Kalk\"ul}
Suppose $g : \bbR \rightarrow \bbR$ is $C^2$.
Write the gradient and Hessian of
\[
  f(x) = g(\|x\|^2)
\]
in terms of derivatives of $g$.

\end{document}
