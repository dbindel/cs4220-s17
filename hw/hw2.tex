\documentclass[12pt, leqno]{article}
\usepackage{amsfonts}
\usepackage{amsmath}
\usepackage{amssymb}
\usepackage{fancyhdr}
\usepackage{hyperref}
\usepackage{tikz}
\usepackage{pgfplots}
\usepackage{listings}

\newcommand{\bbR}{\mathbb{R}}
\newcommand{\bbC}{\mathbb{C}}
\newcommand{\calV}{\mathcal{V}}
\newcommand{\calW}{\mathcal{W}}
\newcommand{\ddiag}{\operatorname{diag}}
\newcommand{\fl}{\operatorname{fl}}
\newcommand{\macheps}{\epsilon_{\mathrm{mach}}}
\newcommand{\matlab}{\textsc{Matlab}}

\newcommand{\hdr}[2]{
  \pagestyle{fancy}
  \lhead{Bindel, Spring 2016}
  \rhead{Numerical Analysis (CS 4220)}
  \fancyfoot{}
  \begin{center}
    {\large{\bf Notes for #1}}
  \end{center}
  \lstset{language=matlab,columns=flexible}  
}


\begin{document}

\hdr{HW 2}

\paragraph{1: Building blocks}
Let $d_1, d_2, u, v \in \bbR^n$ be vectors and define matrices
$D_1 = \operatorname{diag}(d_1)$, $D_2 = \operatorname{diag}(d_1)$,
and $L = uv^T$.  Write an efficient MATLAB or Julia code ($O(n)$ time)
that solves the system
\[
  \begin{bmatrix} D_1 & L \\ 0 & D_2 \end{bmatrix} x = c
\]
The function signature should look like
\begin{lstlisting}{language=matlab,frame=lines,columns=flexible}
% MATLAB/Octave version
function x = hw2solve(d1, d2, u, v, c)

% Julia version (returns a vector x)
function hw2solve(d1, d2, u, v, c)
\end{lstlisting}

\paragraph*{2: Pi, see!}
The following routine estimates $\pi$ by recursively computing the
semiperimeter of a sequence of $2^{k+1}$-gons embedded in the unit circle:
\lstset{language=matlab,frame=lines,columns=flexible}
\lstinputlisting{pibad.m}
Plot the absolute error $|s_k-\pi|$ against $k$ on a semilog plot.
Explain why the algorithm behaves as it does, and describe a
reformulation of the algorithm that does not suffer from this problem.

\paragraph*{3: Low rank limbo}
Suppose $u, v \in \bbR^n$, and let $L = uv^*$.  Show that
\begin{itemize}
\item $\|L\|_1 = \|u\|_1 \, \|v\|_\infty$
\item $\|L\|_\infty = \|u\|_\infty \, \|v\|_1$
\item $\|L\|_F = \|u\|_2 \, \|v\|_2$
\end{itemize}

\end{document}
