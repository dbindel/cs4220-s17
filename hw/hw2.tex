\documentclass[12pt, leqno]{article}
\usepackage{amsfonts}
\usepackage{amsmath}
\usepackage{amssymb}
\usepackage{fancyhdr}
\usepackage{hyperref}
\usepackage{tikz}
\usepackage{pgfplots}
\usepackage{listings}

\newcommand{\bbR}{\mathbb{R}}
\newcommand{\bbC}{\mathbb{C}}
\newcommand{\calV}{\mathcal{V}}
\newcommand{\calW}{\mathcal{W}}
\newcommand{\ddiag}{\operatorname{diag}}
\newcommand{\fl}{\operatorname{fl}}
\newcommand{\macheps}{\epsilon_{\mathrm{mach}}}
\newcommand{\matlab}{\textsc{Matlab}}

\newcommand{\hdr}[2]{
  \pagestyle{fancy}
  \lhead{Bindel, Spring 2016}
  \rhead{Numerical Analysis (CS 4220)}
  \fancyfoot{}
  \begin{center}
    {\large{\bf Notes for #1}}
  \end{center}
  \lstset{language=matlab,columns=flexible}  
}


\begin{document} \hdr{HW 2}{Fri, Feb 12}

\paragraph{1: Building blocks}
Assume $A, B \in \bbR^{n \times n}$ and let {\tt x = solveA(b)} be a function
that efficiently solves the system $Ax = b$.
Write an efficient MATLAB code to solve the system
\[
  \begin{bmatrix} A & B \\ 0 & A \end{bmatrix} x = c
\]

%\lstset
\begin{lstlisting}{language=matlab,frame=lines,columns=flexible}
function x = hw2blockSolve(solveA, B, c)
% Given a solver for systems with a nonsingular matrix A of size
% n x n, a matrix B, also of size n x n, and a column
% vector c of length 2*n, this function solves 
% the system of equations [A B ; zeros(n) A ] x = c
\end{lstlisting}

\paragraph*{2: Numerical stability}
For $x > 1$, the equation $x = \cosh(y)$ can be solved as
\[
  y = -\log\left( x-\sqrt{x^2-1} \right).
\]
Test this formula in MATLAB, Octave, or Python for $x = 10^9$; what
happens?  Rearrange the formula to retain accuracy for $x \gg 1$.

\paragraph*{3: Norms and complements}
Consider the two-by-two block matrix
\[
Z = \begin{bmatrix} A & B \\ C & D \end{bmatrix}
\]
where $D$ and $A$ are square and invertible.  For any of the standard
operator norms (1-norm, 2-norm, $\infty$-norm), show that if
\[
  \|A^{-1}\| \|B\| \|C\| \|D^{-1}\| < 1
\]
then $Z$ is nonsingular and
\[
\|Z^{-1}\| \leq
\frac{\max(\|A^{-1}\|, \|D^{-1}\|)
     (1+\max(\|C\| \|A^{-1}\|, \|B\| \|D^{-1}\|))}
     {1-\|A^{-1}\|\|B\|\|C\|\|D^{-1}\|}.
\]
{\bf Hint:} If $Z^{-1}$ exists, it can be written as
\[
\begin{bmatrix} (I-A^{-1} B D^{-1} C)^{-1} & 0 \\ 0 & (I-D^{-1} C A^{-1} B)^{-1} \end{bmatrix}
\begin{bmatrix} A^{-1} & 0 \\ 0 & D^{-1} \end{bmatrix}
\begin{bmatrix} I & -BD^{-1} \\ -CA^{-1} & I \end{bmatrix}
\]
Use the Neumann series to show that the first term is well defined
under the hypothesis (there is no trouble with the others).  You may
also want to use the fact that for the standard operator norms,
\begin{align*}
\left\| \begin{bmatrix} A & 0 \\ 0 & D \end{bmatrix} \right\| &=
\max(\|A\|, \|D\|) &
\left\| \begin{bmatrix} 0 & B \\ C & 0 \end{bmatrix} \right\| &=
\max(\|B\|, \|C\|).
\end{align*}
This last is not hard to prove, but you do not need to provide the
argument.

\end{document}
