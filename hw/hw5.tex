\documentclass[12pt, leqno]{article}
\input{common}

\begin{document}

\hdr{HW 5}

Recall from HW 4 problem 1 the problem of finding the {\em intersection}
of two cubic Bezier curves.  In this problem, we will find the
{\em closest points} on two cubic Bezier curves $f$ and $g$:
\[
  \min_{t \in [0,1]^2} \|f(t_1)-g(t_2)\|.
\]
\begin{enumerate}
\item
  Write a code to compute the Levenberg-Marquardt step $p$ for a
  given value of the damping parameter $\lambda$:
  \begin{lstlisting}
    function [p] = bezier_lm_step(t, pf, pg, lambda)
  \end{lstlisting}
  If {\tt lambda} is not explicitly provided, your code should default
  to $\lambda = 0$ (a Gauss-Newton step).
\item
  Use Gauss-Newton iteration with line search or Levenberg-Marquardt
  with adaptive $\lambda$ to solve the closest point problem.
\begin{lstlisting}
  function [s,t] = bezier_closest(pf, pg)
  %
  % Compute points s in [0,1] and t in [0,1] such that
  % the distance between f(s) and g(t) is minimized, where f and g
  % are cubic Bezier curves with control points pf and pg (each of
  % dimension d-by-4 with d >= 2).
\end{lstlisting}
  You should {\em not} assume that the closest point is necessarily
  on the interior of the domain; you may deal with the end conditions
  via any reasonable approach, but a barrier or penalty may be
  simplest.  If there are multiple local minima, it is OK to choose one.
\end{enumerate}

\end{document}
