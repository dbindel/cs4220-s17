\documentclass[12pt, leqno]{article} %% use to set typesize 
\input{common}

\begin{document}
\hdr{2017-04-26}

\section{Computing with constraints}

Recall that our basic problem is
\[
\mbox{minimize } \phi(x) \mbox{ s.t. } x \in \Omega
\]
where the feasible set $\Omega$ is defined by equality and inequality
conditions
\[
  \Omega = \{ x \in \bbR^n : c_i(x) = 0, i \in \mathcal{E} \mbox{ and }
    c_i(x) \leq 0, i \in \mathcal{I} \}.
\]
In the last lecture, we described three different ways to formulate
constrained optimization problem that allow us to build on techniques
we previously explored from unconstrained optimization and equation
solving:
\begin{enumerate}
\item {\em Constraint elimination} (for equality constraints):
  Find a parameterization $g : \bbR^{n-m} \rightarrow \Omega$
  formulations and minimize $\phi(g(y))$ without constraints.
  This requires that the constraints be simple (e.g.~affine equality
  constraints).
\item {\em Barriers and penalties:} Add a term to the objective
  function depending on some parameter $\mu$.  This term penalizes $x$
  values that violate the constraint (penalty methods) or that come
  close to $\partial \Omega$ from the inside (barrier methods).  As
  $\mu \rightarrow 0$, the unconstrained minimum of the modified
  problems converges to the constrained minimum of the original.
\item {\em Lagrange multipliers}: Add new variables (multipliers)
  corresponding to ``forces'' needed to enforce the constraints.
  The {\em KKT conditions} are a set of nonlinear equations in the
  original unknowns and the multipliers that characterize constrained
  stationary points.
\end{enumerate}
Our goal now is to sketch how modern constrained optimization
algorithms incorporate these different ways of looking at the problem.
A full treatment is well beyond the scope of the class, but we hope
to give you at least the keywords you will need should you encounter
them in a textbook, paper, or a cocktail party\footnote{If you must,
  replace ``cocktail party'' with ``job interview'' --- but if you do,
  I think you should seek more interesting cocktail parties.  No, I do
  not get invited to many cocktail parties myself.}.
Ideally, knowing something about what happens in the algorithms will
also help you think about which of various equivalent formulations
of an optimization problem will be more (or less) friendly to solvers.
The plan is to first give a ``lay of the land'' of different families
of algorithms, then to give a more detailed treatment with the running
example of linearly constrained quadratic programs.

\section{Lay of the Land}

As we mentioned before, problems with {\em inequality} constraints
tend to be more difficult than problems with {\em equality}
constraints alone, because it involves the combinatorial subproblem
of figuring out which constraints are {\em active}
(a constraint $c_i(x) \leq 0$ is active if $c_i(x) = 0$ at the optimum).
Once we have figured out the set of active constraints, we can reduce
an inequality-constrained problem to an equality-constrained problem.
Hence, the purely equality-constrained case is an important subproblem
for inequality-constrained optimizers, as well as a useful problem
class in its own right.

For problems with only equality constraints, there are several
standard options:
\begin{itemize}
\item
  {\em Null space methods} deal with linear equality constraints
  by reducing to an unconstrained problem in a lower-dimensional space.
\item
  {\em Projected gradient methods} deal with simple equality
  constraints by combining a (scaled) gradient step and a projection
  onto a constraint set.
\item
  {\em Penalty methods} approximately solve an equality-constrained
  problem through an unconstrained problem with an extra term that
  penalizes proposed soutions that violate the constraints.  That is,
  we use some constrained minimizer to solve
  \[
    \mbox{minimize } \phi(x) + \frac{1}{\mu} \sum_{i \in\mathcal{E}} c_i(x)^2.
  \]
  As $\mu \rightarrow 0$, the minimizers to these approximate problems
  approach the true minimizer, but the Hessians that we encounter
  along the way become increasingly ill-conditioned (with condition
  number proportional to $\mu^{-1}$).
\item
  {\em KKT solvers} directly tackle the first-order optimality
  conditions (the KKT conditions), simultaneously computing the
  constrained minimizer and the associated Lagrange multipliers.
\item
  {\em Augmented Lagrangian} methods combine the advantages of penalty
  methods and the advantages of the penalty formulation.  In an
  augmented Lagrangian solver, one finds critical points for the
  augmented Lagrangian
  \[
  \mathcal{L}(x, \lambda; \mu) =
    \phi(x) + \frac{1}{\mu} \sum_{i \in \mathcal{E}} c_i(x)^2 + \lambda^T c(x)
   \]
  by alternately adjusting the penalty parameter $\mu$ and the
  Lagrange multipliers.
\end{itemize}

In the inequality-constrained case, we have
\begin{itemize}
\item {\em Active set methods} solve (or approximately solve) a
  sequence of equality-constrained subproblems, shuffling constraints
  into and out of the proposed working set along the way.  These
  methods are particularly attractive when one has a good initial
  estimate of the active set.
\item
  {\em Projected gradient methods} deal with simple inequality
  constraints by combining a (scaled) gradient step and a projection
  onto a constraint set.
\item
  {\em Barrier methods} and {\em penalty methods} add a term to the
  objective function in order to penalize constraint violations or
  near-violations; as in the equality-constrained case, a parameter
  $\mu$ governs a tradeoff between solution quality and conditioning
  of the Hessian matrix.
\item
  {\em Interior point methods} solve a sequence of barrier subproblems
  using a continuation strategy, where the barrier or penalty
  parameter $\mu$ is the continuation parameter.  This is one of the
  most popular modern solver strategies, though active set methods may
  show better performance when one ``warm starts'' with a good initial
  guess for the solution and the active set of constraints.
\end{itemize}
As with augmented Lagrangian strategies in the equality-constrained
case, state-of-the art strategies for inequality-constrained problems
often combine approaches, using continuation with respect to a
barrier parameters as a method of determining the active set of
constraints in order to get to an equality-constrained subproblem with
a good initial guess for the solution and the Lagrange multipliers.

The {\em sequential quadratic programming} (SQP) approach for nonlinear
optimization solves a sequence of linearly-constrained quadratic
optimization problems based on Taylor expansion of the objective and
constraints about each iterate.  This generalizes simple Newton
iteration for unconstrained optimization, which similarly solves a
sequence of quadratic optimization problems based on Taylor expansion
of the objective.  Linearly-constrained quadratic programming problems
are hence an important subproblem in SQP solvers, as well as being an
important problem class in their own right.

\section{Quadratic programs with equality constraints}

% Basic setup
% Variables sum to one
% Problems with general sparsity

We begin with a simple case of a quadratic objective and linear
equality constraints:
\begin{align*}
  \phi(x) &= \frac{1}{2} x^T H x - x^T d \\
  c(x) &= A^T x-b = 0,
\end{align*}
where $H \in \bbR^{n \times n}$ is symmetric and positive definite
{\em on the null space of $A^T$} (it may be indefinite or singular overall),
$A \in \bbR^{n \times m}$ is full rank with $m < n$, and $b \in \bbR^m$.
Not only are such problems useful in their own right, solvers for
these problems are also helpful building blocks for more sophisticated
problems --- just as minimizing an unconstrained quadratic can be seen
as the starting point for Newton's method for unconstrained optimization.

\subsection{Constraint elimination (linear constraints)}

As discussed last time, we can write the space of solutions to the
constraint equations in terms of a (non-economy) QR decomposition
of $A$:
\[
  A =
  \begin{bmatrix} Q_1 & Q_2 \end{bmatrix}
  \begin{bmatrix} R_1 \\ 0 \end{bmatrix}
\]
where $Q_2$ is a basis for the null space of $A$.  The set of
solutions satisfying the constraints $Ax = b$ is
\[
  \Omega = \{ u + Q_2 y : y \in \bbR^{(n-m)}, u = Q_1 R_1^{-T} b \};
\]
here $u$ is a {\em particular solution} to the problem.  If we
substitute this parameterization of $\Omega$ into the objective,
we have the unconstrained problem
\[
  \mbox{minimize } \phi(u + Q_2 y).
\]
While we can substitute directly to get a quadratic objective in
terms of $y$, it is easier (and a good exercise in remembering
the chain rule) to compute the stationary equations
\begin{align*}
  0
  &= \nabla_y \phi(u + Q_2 y) 
  = \left(\frac{\partial x}{\partial y}\right)^T \nabla_x \phi(u+Q_2 y) \\
  &= Q_2^T (H (Q_2 y + u) - d) 
  = (Q_2^T H Q_2) y - Q_2^T (d-Hu).
\end{align*}
In general, even if $A$ is sparse, $Q_2$ may be dense, and so even if
$H$ is dense, we find that $Q_2^T H Q_2$ is dense.

\subsection{Barriers, penalties, and conditioning}

Now consider a penalty formulation of the same equality-constrained
optimization function, where the penalty is quadratic:
\[
  \mbox{minimize } \phi(x) + \frac{1}{2\mu} \|A^T x-b\|^2.
\]
In fact, the augmented objective function is again quadratic, and
the critical point equations are
\[
  (H + \mu^{-1} AA^T) x = d + \mu^{-1} A b.
\]
We can analyze this more readily by changing to the $Q$ basis from
the QR decomposition of $A$ that we saw in the constraint elimination
approach:
\[
\begin{bmatrix}
  Q_1^T H Q_1 + \mu^{-1} R_1 R_1^T & Q_1^T H Q_2 \\
  Q_2^T H Q_1 & Q_2^T H Q_2
\end{bmatrix}
(Q^T x) =
\begin{bmatrix}
  Q_1^T d + \mu^{-1} R_1 b \\
  Q_2^T d
\end{bmatrix}
\]
Taking a Schur complement, we have
\[
(\mu^{-1} R_1 R_1^T + F)(Q_1^T x) = \mu^{-1} R_1 b - g
\]
where
\begin{align*}
  F &= Q_1^T H Q_1 - Q_1^T H Q_2 (Q_2^T H Q_2)^{-1} Q_2^T H Q_1 \\
  g &= [I - Q_1^T H Q_2 (Q_2^T H Q_2)^{-1} Q_2^T] d
\end{align*}
As $\mu \rightarrow 0$, the first row of equations is dominated by the
$\mu^{-1}$ terms, and we are left with
\[
  R_1 R_1^T (Q_1^T x) - R_1 b \rightarrow 0
\]
i.e.~$Q_1 Q_1^T x$ is converging to $u = Q_1 R_1^{-T} b$, the
particular solution that we saw in the case of constraint elimination.
Plugging this behavior into the second equation gives
\[
  (Q_2^T H Q_2) (Q_2^T x) - Q_2^T (d-Hu) \rightarrow 0,
\]
i.e.~$Q_2^T x$ asymptotically behaves like $y$ in the previous
example.  We need large $\mu$ to get good results if the constraints
are ill-posed or if $Q_2^T H Q_2$ is close to singular.  But in
general the condition number scales like $O(\mu^{-1})$, and so large
values of $\mu$ correspond to problems that are numerically
unattractive.

\subsection{Lagrange multipliers and KKT systems}

% Basic setup -- same as before
% Enforcing constraints (quadratic penalty)
% Limiting behavior

The KKT conditions for our equality-constrained problem say that the
gradient of
\[
  L(x,\lambda) = \phi(x) + \lambda^T (A^T x-b)
\]
should be zero.  In matrix form, the KKT system (saddle point system)
\[
  \begin{bmatrix}
    H & A \\
    A^T & 0
  \end{bmatrix}
  \begin{bmatrix} x \\ \lambda \end{bmatrix} =
  \begin{bmatrix} d \\ b \end{bmatrix}.
\]
If $A$ and $H$ are well-conditioned, then so is this system,
so there is no bad numerical behavior.  The system also retains
whatever sparsity was present in the original system matrices
$H$ and $A$.  However, adding the Lagrange multipliers not only
increases the number of variables, but the extended system lacks
any positive definiteness that $H$ may have.

The KKT system is closely related to the penalty formulation that
we saw in the previous subsection, in that if we use Gaussian
elimination to remove the variable $\lambda$ in
\[
  \begin{bmatrix}
    H & A \\
    A^T & -\mu I
  \end{bmatrix}
  \begin{bmatrix} \hat{x} \\ \lambda \end{bmatrix} =
  \begin{bmatrix} d \\ b \end{bmatrix},
\]
we have the Schur complement system
\[
  (H+\mu^{-1} AA^T) \hat{x} = d + \mu^{-1} A b,
\]
which is identical to the stationary point condition for the
quadratically penalized objective.

\subsection{Augmented Lagrangian}

From a solver perspective, the block 2-by-2 structure of the KKT
system looks highly attractive.  Alas, we do {\em not} require that
$H$ be positive definite, nor even that it be nonsingular; to have a
unique global minimum, we only need positive definiteness of the
projection of $H$ onto the null space (i.e.~$Q_2^T H Q_2$ should be
positive definite).  This means we cannot assume that (for example)
$H$ will admit a Cholesky factorization.

The augmented Lagrangian approach can be seen as solving the
constrained system
\[
  \mbox{minimize } \frac{1}{2} x^T H x - d^T x + \frac{1}{2\mu} \|A^T x-b\|^2
  \mbox{ s.t. } Ax = b.
\]
The term penalizing nonzero $\|Ax-b\|$ is, of course, irrelevant at
points satisfying the constraint $Ax = b$.  Hence, the constrained
minimum for this augmented objective is identical to the constrained
minimum of the original objective.  However, if the KKT conditions for
the modified objective take the form
\[
  \begin{bmatrix}
    H+\mu^{-1}AA^T & A \\
    A^T & 0
  \end{bmatrix}
  \begin{bmatrix} x \\ \lambda \end{bmatrix} =
  \begin{bmatrix} d + \mu^{-1} A b \\ b \end{bmatrix}.
\]
Now we do not necessarily need to drive $\mu$ to zero to obtain a good
solution; but by choosing $\mu$ small enough, we can ensure that
$H + \mu^{-1} AA^T$ is positive definite (assuming that the problem is
convex subject to the constraint).

\end{document}
