\documentclass[12pt, leqno]{article} %% use to set typesize 
\usepackage{amsfonts}
\usepackage{amsmath}
\usepackage{amssymb}
\usepackage{fancyhdr}
\usepackage{hyperref}
\usepackage{tikz}
\usepackage{pgfplots}
\usepackage{listings}

\newcommand{\bbR}{\mathbb{R}}
\newcommand{\bbC}{\mathbb{C}}
\newcommand{\calV}{\mathcal{V}}
\newcommand{\calW}{\mathcal{W}}
\newcommand{\ddiag}{\operatorname{diag}}
\newcommand{\fl}{\operatorname{fl}}
\newcommand{\macheps}{\epsilon_{\mathrm{mach}}}
\newcommand{\matlab}{\textsc{Matlab}}

\newcommand{\hdr}[2]{
  \pagestyle{fancy}
  \lhead{Bindel, Spring 2016}
  \rhead{Numerical Analysis (CS 4220)}
  \fancyfoot{}
  \begin{center}
    {\large{\bf Notes for #1}}
  \end{center}
  \lstset{language=matlab,columns=flexible}  
}


\begin{document}
\hdr{2016-04-13}

\section*{Broadening the Basin}

If $f(x^*) = 0$ is a solution to a nonlinear equation, the {\em basis
  of convergence} for $x^*$ for a given iteration is the set of
initial guesses for which the iteration converges to $x^*$.  In the
previous lecture, we gave some indication of features that can lead to
small convergence basins for Newton-like iterations: we expect
problems if $f'$ is nearly singular in the vicinity of $x^*$, or if
$f'$ can change rapidly (i.e.~it is not controlled by a modest
Lipschitz constant).  One way to deal with this problem is to find
a very good initial guess; another approach, known
as {\em globalization}, changes the iteration in simple ways
in order to expand the basin of convergence.  Globalization does not
free us from getting a good initial guess, but it does make the
quality of the initial guess slightly less crucial to guaranteeing
convergence.

So far, we focused on nonlinear equation solving.  To understand
globalization, though, it will help to focus on optimization problems.

\section*{All Downhill from Here}

Let $\phi : \bbR^n \rightarrow \bbR$ be a $C^1$ function.
At a point $x \in \bbR^n$, a vector $0 \neq p \in \bbR^k$
points in a {\em descent} direction if
\[
  \phi'(x) p < 0.
\]
If we move from $x$ by a small amount in a descent direction, we
decrease the function value; that is, for small enough $\epsilon$,
\[
  \phi(x+\epsilon p)
  = \phi(x) + \epsilon \phi'(x) p + o(\epsilon)
  < \phi(x).
\]
The most familiar descent direction is the {\em steepest descent}
direction $-\nabla \phi$.  More generally, if $A$ is a positive
definite matrix, then $p = -A \nabla \phi(x)$ is a descent direction
whenever $x$ is not a stationary point, since
\[
  \phi'(x) p = -u^T A u, \quad u = \nabla \phi(x)
\]
and positive definiteness guarantees $u^T A u > 0$.  In particular,
if $x$ is near a strong local minimum and $\phi$ is $C^2$, then
we expect the Hessian to be positive definite, so that the Newton
update $p = -H(x)^{-1} \nabla \phi(x)$ is a descent direction.

Of course, sometimes the Hessian could be indefinite or negative
definite!  In this case, even moving a little in the Newton direction
could increase $\phi$.  We would usually consider this a Bad Thing.
Therefore, in globalized optimization methods, we usually consider
steps that are proportional to
\[
  p = -\hat{H}^{-1} \nabla \phi(x)
\]
where $\hat{H}$ is some positive definite matrix.  If $\hat{H} = I$,
this gives us the steepest descent direction.  For globalized Newton
methods, we let $\hat{H}$ be equal to the Hessian matrix when the
Hessian is sufficiently positive definite, and otherwise take
$\hat{H}$ to be some positive definite matrix.

What if we are interested not in optimization, but in finding zeros
of a nonlinear function $f : \bbR^n \rightarrow \bbR^n$?
Recall that finding zeros of $f$ is equivalent to finding
minima of $\phi(x) = \|f(x)\|^2$, for which descent directions satisfy
\[
  \phi'(x) p = 2 f(x)^T f'(x) p < 0.
\]
If the Newton direction $p = -f'(x)^{-1} f(x)$ is well
defined and nonzero, it satisfies
\[
  \phi'(x) p = -2 \|f(x)\|^2,
\]
and so the Newton step for $f$ is certainly in a descent direction
for $\phi$.

\section*{Line Up!}

If $p \in \bbR^n$ is a descent direction for $\phi$, then we know
\[
  \phi(x + \epsilon p) = \phi(x) + \epsilon \phi'(x) p + o(\epsilon) < \phi(x)
\]
for small enough $\epsilon$.  We can be even sharper and say that for
any $\eta < 1$ and for small enough $\epsilon$, we have
\[
  \phi(x+\epsilon p) \leq \phi(x) + \epsilon \eta \phi'(x) p.
\]
Alas, unless we have more information (e.g.~a Lipschitz constant for
the derivative of $\phi$), we cannot tell in advance how small is
``small enough.''

In a {\em line search} procedure, we consider first choose a descent
direction $p$ and then consider new points of the form $x + \sigma p$
for some scalar $\sigma$.  A natural choice is to consider {\em
  backtracking}: we look at points $x + \alpha^l p$ where
$0 < \alpha < 1$ (a typical choice is $\alpha = 0.5$) and choose the
smallest $l \geq 0$ such that
\[
  \phi(x + \alpha^l p) \leq \phi(x) + \eta \alpha^l \phi'(x) p.
\]
An alternative to this geometric line search is to use {\em exact}
line search: that is, find $s$ to minimize $f(x+sp)$ along the ray.
In practice, the cost of an exact line search is rarely worthwhile.

A successful {\em monotone} line search means that we will move from
one guess at the minimizer to a new guess with a lower objective
value.  Moreover, if $p$ is a descent direction, then we should always
be able to conduct the line search successfully.  There is, however, a
caveat.  Even putting aside the possibility that a programming error
will lead us to propose a step in a non-descent direction (which we
can and should guard against by careful programming), we might find
that a proposed direction is simply not a very {\em good} descent
direction, either because it is nearly orthogonal to the gradient or
because the function is a little crazy.  In either case, we might find
ourselves cutting the step by what seems like an unreasonable amount.
In the worst case, we might find that $|\alpha^l p| < |x|$
componentwise, so that $\fl(x+\alpha^l p) = \fl(x)$; in this case, we
keep trying the same value over and over again, and end up in an
infinite loop.  We guard against this possibility in two ways.  First,
we will make sure that we bound the number of steps that we allow in a
line search --- a good idea for any iterative procedure!  Second, we
will try to rule out directions that form too acute an angle with the
gradient, and thus are not ``sufficiently downhill'' for the algorithm
to make acceptable progress.

There has also been some work on {\em non-monotone} line search
algorithms that allow increases in the function values, as long as
progress is made in some more averaged sense (e.g.~the new point
has a objective function value smaller than the maximum objective
function for the past few points).  This is useful for improving
convergence speed on some hard problems, and is useful in the context
of particular classes of methods such as spectral projected gradient
(about which we will say nothing in this class other than the name).

\section*{Decent Descent}

Choosing descent directions and employing line search is enough
to strongly suggest convergence, but there are still some pathological
ways to get into trouble.  In particular:
\begin{itemize}
\item Our proposed steps $p^k$ might blow up.  To avoid this, we will
  usually insist on algorithms where $\|p^k\|$ remains bounded for all
  $k$.
\item Our proposed steps $p^k$ might shrink too quickly, so that we
  converge before actually reaching a minimizer.  To avoid this,
  we insist on algorithms where $\|p^k\| \geq m \|\nabla \phi(x^k)\|$
  for all $k$ (such steps are called {\em gradient related}).
\item Our proposed steps might remain a reasonable length, but come
  closer and closer to orthogonal to the gradient vector, so that
  progress slows too quickly for us to converge to a minimizer.
\end{itemize}

In general, if our objective is $C^1$ with a Lipschitz first
derivative, and if we use the line search algorithm sketched above
(backtracking line search where we pick the longest step that gives
sufficient decrease) and a sequence of proposed steps $p^k$ that
are gradient related, form acute angles with the gradient that are
bounded away from right angles, then we are guaranteed to converge
to a stationary point whenever the set of points less than the initial
value $\phi(x^0)$ is bounded.  Note that this last condition is
important!  For instance, a function like
\[
  \phi(x) = \exp(-x^2/2)-exp(-x^4/2)
\]
has two global minima close to zero, but for any starting guess
outside the interval $[-3,3]$ (or even a bit more), any iteration
based on descent directions can only move {\em away} from those
mimina.  Also, we are only guaranteed to converge to a stationary
point; it could be a saddle, or a local minimizer that is not the one
we care about.  For all of these reasons, good initial guesses remain
important even with globalization.

Newton's method has many attractive properties, particularly when we
combine it with a globalization strategy.  Unfortunately, Newton steps
are not cheap.  At each step, we need to:
\begin{itemize}
\item Form the function $f$ {\em and} the Jacobian.  This involves not
  only computational work, but also analytical work -- someone needs
  to figure out those derivatives!
\item Solve a linear system with the Jacobian.  This is no easier than
  any other linear solve problem!  Indeed, it may be rather expensive
  for large systems, and factorization costs cannot (in general) be
  amortized across Newton steps.
\end{itemize}
The Jacobian (or the Hessian if we are looking at optimization
problems) is the main source of difficulty.  Today we consider several
iterations that deal with this difficulty in one way or the other.

\section*{Trust but Verify}

The line search paradigm for globalized optimization involves
two phase:
\begin{itemize}
\item Propose a step in a descent direction (with some conditions
  mentioned above).
\item Potentially cut the step back to get sufficient decrease.
\end{itemize}
But if we choose a step according to Newton's method, there is
something odd about this two-stage process.  Why should we choose
a direction based on a model that we think will be good globally,
then keep using that direction after we find out it was not a
good direction?  An alternative approach is to define a
{\em trust region} where we believe the model to be good,
and choose the next iterate to lie within the trust region.
That is, at each step we minimize an approximation
\[
  f(x^k+p^k) \approx f(x^k) + f'(x^k) p^k + \frac{1}{2} (p^k)^T H(x^k) p^k
\]
subject to the constraint that $\|p^k\| \leq \rho$.  This leads to the
subproblem
\[
  (H+\lambda I) p^k = -\nabla f(x^k)
\]
where $\lambda$ is a multiplier that enforces the constraint $\|p^k\|
\leq \rho$.  We then accept or reject the point, and dynamically
adjust $\rho$, depending on how well the objective value at the new
point agrees with the model.

Trust region methods are generally more complex to implement than line
search methods, particularly since one often approximates the
constrained minimization problem at the heart of the method.  On the
other hand, there are sometimes reasons to prefer trust region
approaches.

\section*{Gauss-Newton and Levenberg-Marquardt}

The trust region approach was first used in the context of nonlinear
least squares problems, where the {\em Gauss-Newton} iteration
involves
\[
  \min_{p^k} \|f(x^k) + f'(x^k) p^k\|.
\]
The Gauss-Newton iteration is {\em not} the same as Newton,
since the Hessian of $\phi(x) = \|f(x)\|^2/2$ is the matrix
with elements
\[
  \phi_{,ij}(x) = \sum_k [f_{k,i}(x) f_{k,j}(x) + f_{k}(x) f_{k,ij}(x)],
\]
or, equivalently,
\[
  H(x) \equiv \nabla^2 \phi(x) = f'(x)^T f'(x) + \sum_{k} f_k(x) \nabla^2 f_k(x).
\]
The first term is the matrix that appears in the Gauss-Newton step,
but Gauss-Newton lacks the second term.  On the other hand, if the
residual is small at the minimizer, the second term often contributes
only a little.

The {\em Levenberg-Marquardt} iteration can be seen as a regularized
version of the Gauss-Newton iteration, i.e.
\[
  \min_{p^k} \|f(x^k) + f'(x^) p^k\| + \lambda_k \|p^k\|.
\]
In Levenberg-Marquardt, one typically works with $\lambda_k$ directly,
rather than treating it as a multiplier that enforces a constraint
that $\|p^k\|$ lie in some bounded domain.  However, it is possible to
treat the method either from the perspective of trust regions or
from the perspective of regularization.

\end{document}
