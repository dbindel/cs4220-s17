\documentclass[12pt, leqno]{article} %% use to set typesize 
\usepackage{amsfonts}
\usepackage{amsmath}
\usepackage{amssymb}
\usepackage{fancyhdr}
\usepackage{hyperref}
\usepackage{tikz}
\usepackage{pgfplots}
\usepackage{listings}

\newcommand{\bbR}{\mathbb{R}}
\newcommand{\bbC}{\mathbb{C}}
\newcommand{\calV}{\mathcal{V}}
\newcommand{\calW}{\mathcal{W}}
\newcommand{\ddiag}{\operatorname{diag}}
\newcommand{\fl}{\operatorname{fl}}
\newcommand{\macheps}{\epsilon_{\mathrm{mach}}}
\newcommand{\matlab}{\textsc{Matlab}}

\newcommand{\hdr}[2]{
  \pagestyle{fancy}
  \lhead{Bindel, Spring 2016}
  \rhead{Numerical Analysis (CS 4220)}
  \fancyfoot{}
  \begin{center}
    {\large{\bf Notes for #1}}
  \end{center}
  \lstset{language=matlab,columns=flexible}  
}


\begin{document}
\hdr{2016-04-08}

\section*{Life beyond Newton}

Newton's method has many attractive properties, particularly when we
combine it with a globalization strategy.  Unfortunately, Newton steps
are not cheap.  At each step, we need to:
\begin{itemize}
\item Form the function $f$ {\em and} the Jacobian.  This involves not
  only computational work, but also analytical work -- someone needs
  to figure out those derivatives!
\item Solve a linear system with the Jacobian.  This is no easier than
  any other linear solve problem!  Indeed, it may be rather expensive
  for large systems, and factorization costs cannot (in general) be
  amortized across Newton steps.
\end{itemize}
The Jacobian (or the Hessian if we are looking at optimization
problems) is the main source of difficulty.  Today we consider several
iterations that deal with this difficulty in one way or the other.

\section*{Almost Newton analysis}

In these notes, we will be somewhat careful about the analysis, but in
general you are {\em not} responsible for remembering this level of
detail.  We will try to highlight the points that are important in
practice for understanding when solvers might run into trouble, and why.

A common theme in our analysis of ``almost Newton'' iterations is that
we will build on Newton convergence.  In order to simplify life, we
assume throughout that $f$ is $C^1$ and the Jacobian is Lipschitz with
constant $M$ and has a bounded inverse (i.e.~$\|f'(x)^{-1}\| \leq K$
for $x$ in some neighborhood of $x^*$).

Let's start by looking again at the error from a Newton step.
A Newton step starting from $x$ near $x^*$ is
\[
  p(x) = -f'(x)^{-1} f(x) = -f'(x)^{-1} [f(x)-f(x^*)].
\]
By the mean value theorem, there is a point
$\tilde{x} = x^* + \xi (x-x^*)$ on the line segment between
$x$ and $x^*$ such that
\[
  f(x)-f(x^*) = f'(\tilde{x}) (x-x^*).
\]
Thus, we have
\begin{align*}
  p(x)
  &= -f'(x)^{-1} f'(\tilde{x}) (x-x^*) \\
  &= -(x-x^*) + f'(x)^{-1} [f'(x)-f'(\tilde{x})](x-x^*)
  &= -(x-x^*) + d(x).
\end{align*}
Applying the Lipschitz condition and bounded inverse assumption,
\[
  \|d(x)\| \leq BM \|x-\tilde{x}\|\|x-x^*\| \leq BM \|x-x^*\|^2.
\]
Therefore,
\[
  x + p(x) = x^* + d(x)
\]
and $d(x) = O(\|x-x^*\|^2)$; that is, the iteration
$x \mapsto x + p(x)$ converges quadratically from starting points
near enough $x^*$.

We can also see that a sufficient condition for convergence is that
the initial error is less than $1/(BM)$.  This differs from our
earlier bound of $2/(3BM)$ only because we assumed a uniform bound on
the inverse of the Jacobian in the relevant region rather than
assuming a bound at the solution and using the Lipschitz behavior to
get everything else.

Now suppose we have an iteration
\[
  x^{k+1} = x^k + \hat{p}^k
\]
where $\hat{p}^k$ is an approximation to the Newton step $p(x^k)$.
Subtracting $x^*$ from both sides and adding $p(x^k)-p(x^k)$ to the
right side gives
\[
  e^{k+1} = e^k + p(x^k) - (p(x^k)-\hat{p}^k),
\]
and taking norms gives
\[
  \|e^{k+1}\| \leq BM \|e^k\|^2 + \|p(x^k)-\hat{p}^k\|.
\]
Therefore, we can think of our convergence analysis in two steps:
we first analyze the error in the Newton iteration, then analyze
how close our approximate Newton step is to a true Newton step.

In general, to get linear convergence, we need that
$\|p(x^k)-\hat{p}^k\| < \alpha \|e^k\|$ for some $\alpha < 1$.  If we
have this condition, then the approximate Newton iteration will
converge whenever $\|e^0\| \leq (1-\alpha)/BM$.

\section*{Approximate Jacobians}

We start with a family of iterations $x^{k+1} = x^k + \hat{p}^k$ where
\[
  (f'(x^k)+E_k) \hat{p}^k = -f(x^k).
\]
That is, the error in the step comes from a small error in the
Jacobian.  As in our analysis of Newton, we replace the right hand
side using the mean value theorem
\[
  (f'(x^k)+E_k) \hat{p}^k = -f'(\tilde{x})(x^k-x^*).  
\]
where $\tilde{x}$ lies on the segment between $x^k$ and $x^*$.
Rearranging as before, we have
\[
  \hat{p}^k =
    -e^k +
    \left( f'(x^k)+E_k)^{-1} \right)^{-1}
    \left(f'(x^k) + E - f'(\tilde{x}) \right) e^k.
\]
Therefore, the error iteration $e^{k+1} = e^k + \hat{p}^k$ becomes
\[
  e^{k+1} =
    \left( f'(x^k)+E_k)^{-1} \right)^{-1}
    \left(f'(x^k) + E_k - f'(\tilde{x}) \right) e^k.
\]
Taking several bounds gives us (for $B \|E_k\| < 1$),
\[
  \|e^{k+1}\| \leq \left( \frac{BM\|e^k\| + \|E^k\|}{1-B\|E_k\|} \right) \|e^k\|
  = O(\|e^k\|^2) + O(\|e^k\| \|E^k\|).
\]
The constant in parentheses is less than one if
\[
  \|e^k\| \leq \frac{1-(B+1)\|E^k\|}{BM},
\]
which for the zero error case reverts to the bound we saw before.

How should we read these bounds?  If the bound on $B$ is large
(i.e.~$f'$ can be very nearly singular), then the method can blow up
either if the initial error is too big or if $\|E^k\|$ can get big;
after all, for $\|E^k\| > B^{-1}$ the system for an approximate step
might be singular!  We can also see issues with the rate of
convergence if $f'$ moves around quickly (i.e.~$M$ is large);
but this is an issue with Newton iteration as well, and switching to
an approximate Newton iteration makes things neither better nor worse.

\subsection*{Chord iteration}

The {\em chord iteration} is
\[
  x^{k+1} = x^k - f'(x^0)^{-1} f(x^k).
\]
Written in this way, the method differs from Newton in only one
character --- but what a difference it makes!  By re-using the
Jacobian at $x^0$ for all steps, we degrade the progress per step, but
each step becomes cheaper.  In particular, we can benefit from
re-using a factorization across several steps:
\begin{lstlisting}
  % Given a function [f,J] = f(x) that returns the Jacobian only if
  % it is used as a second argument, run a chord iteration starting
  % from some initial x0
  function [x] = chord_solve(f, x0, nsteps)

    [f0,J0] = f(x0);
    [L,U,P] = lu(J0);
    x = x0;
    for k = 1:nsteps
      fx = f(x);
      p = - U\(L\(P*fx);
      x = x + p;
    end
\end{lstlisting}

In terms of the approximate Newton framework, the chord iteration
involves errors $\|E_k\| = \|f'(x^k)-f'(x^0)\| \leq M \|e^0\|$.
Therefore, the iteration is guaranteed to converge for starting
points such that $\|e^0\| < 1/(2BM+M)$, and the error in successive
iterates is bounded by.
\[
  \|e^{k+1}\| \leq \left( \frac{BM\|e^k\| + M \|e^0\|}{1-BM\|e^0\|}
  \right) \|e^k\| = O(\|e^0\| \|e^k\|).
\]

\subsection*{Shamanskii iteration}

The chord method involves using one approximate Jacobian forever.
The Shamanskii method involves freezing the Jacobian for $m$ steps
before getting a new Jacobian; that is, one step of Shaminskii looks
like
\begin{align*}
  x^{k+1,0} & = x^k \\
  x^{k+1,j+1} &= x^{k+1,j} - f'(x^k)^{-1} f(x^{k+1,j}) \\
  x^{k+1} &= x^{k+1,m}.
\end{align*}
Like the chord iteration, Shaminskii is guaranteed to converge for
starting points such that $\|e^0\| < 1/(2BM+M)$.  The error for
each iteration (from $x^k$ to $x^{k+1}$, not from $x^{k+1,j}$ to
$x^{k+1,j+1}$) satisfies
\[
  \|e^{k+1}\|
  \leq \left( \frac{BM+M}{1-BM\|e^k\|} \right) \|e^{k}\|^{m+1}
  = O(\|e^k\|^{m+1}).
\]
Beyond the chord and Shaminskii iterations, the idea of re-using
Jacobians occurs in several other methods.

\subsection*{Finite-difference Newton}

So far, we have assumed that we can compute the Jacobian if we want
it.  What if we just don't want to do the calculus to compute
Jacobians?  A natural idea is to approximate each column of the
Jacobian by a finite difference estimate:
\[
  f'(x^k) e_j \approx \frac{f(x^k+he_j)-f(x^k)}{h}
\]
Using Lipschitz bounds on $f'$ gives the error bound
\[
  \left\| f'(x^k) - \frac{f(x^k+he_j)-f(x^k)}{h} \right\| \leq Mh,
\]
and an approximation to $f'(x^k)$ based on finite difference
approximation would have a two norm error of at most
$\|E^k\| \leq \sqrt{n} M h$.  The method must converge for
initial errors less than $(1-\sqrt{n} (B+1) Mh)/(1-\sqrt{n} BMh)$,
and the convergence is bounded by
\[
  \|e^{k+1}\| \leq
  \left( \frac{BM\|e^k\| + \sqrt{n}Mh}{1-\sqrt{n} BMh} \right) \|e^k\| =
  O(h\|e^k\|).
\]

\section*{Inexact Newton}

So far, we have considered approximations to the Newton step based on
approximation of the Jacobian matrix.  What if we instead used the
exact Jacobian matrix, but allowed the update linear systems to be solved
using an iterative solver?  In this case, there is a small residual,
i.e.
\[
  f'(x^k) \hat{p}^k = -f(x^k) + r^k
\]
where $\|r^k\| \leq \eta_k \|f(x^k)\|$ (i.e.~$\eta_k$ is a relative
residual tolerance on the solve).  In this case,
\[
  \|\hat{p}^k-p(x^k)\| = \|f'(x^k)^{-1} r^k\| \leq B \|r^k\| \leq
  \eta_k B \|f(x^k)\|.
\]
We also have that
\[
  \|f(x^k)\| = \|f(x^k)-f(x^*)\| = \|f'(\tilde{x}) e^k\| \leq C \|e^k\|
\]
where $C$ is a bound on the norm of $f'$.  Thus
\[
  \|\hat{p}^k-p(x^k)\| \leq \eta_k BC \|e^k\|,
\]
which we combine with the bound from the start of the notes to give
\[
\|e^{k+1}\| \leq B(M \|e^k\| + \eta_k C) \|e^k\|
  = O(\|e^k\|^2) + O(\eta_k \|e^k\|).
\]
Hence, we have the following trade-off.  If we solve the systems very
accurately ($\eta_k$ small), then inexact Newton will behave much
like ordinary Newton.  Thus, we expect to require few steps of the
outer, nonlinear iteration; but the inner iteration (the linear
solver) may require many steps to reach an acceptable residual
tolerance.  In contrast, if we choose $\eta_k$ to be some modest
constant independent of $k$, then we expect linear convergence of the
outer nonlinear iteration, but each step may run relatively fast,
since the linear systems are not solved to high accuracy.

One attractive feature of Krylov subspace solvers for the Newton
system is that they only require matrix-vector multiplies with
the Jacobian --- also known as directional derivative computations.
We can approximate these directional derivaties by finite differences
to get a method that may be rather more attractive than computing
a full Jacobian approximation by finite differencing.  However,
it is necessary to use a Krylov subspace method that tolerates inexact
matrix vector multiplies (e.g.~FGMRES).

\section*{Broyden}

A {\em quasi-Newton} method updates approximations to both the
solution and the Jacobian over the course of an iteration.  The
most famous such method, Broyden's method, is analogous to the secant
iteration in 1D.  The iteration is
\[
  x^{k+1} = x^k - B_k^{-1} f(x^k)
\]
where $B_k$ is computed by the update
\[
  B_{k+1} = B_k + \frac{f(x^{k+1}) (p^k)^T}{\|p^k\|^2}
\]
where $p^k = x^{k+1}-x^k$.  This update satisfies the {\em secant equation}
\[
  B_{k+1} p^k = f(x^{k+1})-f(x^k).  
\]
It is not the only solution to the secant equation, but it is a good
one.

Broyden's method converges superlinearly, though not quadratically;
the argument is beyond the scope of these notes, though none of the
ideas are that complicated.  For small problems, the implementations
of Broyden are interesting in that they involve continuously updating
factorizations to account for new low-rank additions to an approximate
Jacobian.  This can be done in $O(n^2)$ time (rather than $O(n^3))$ if
one is clever about the numerical linear algebra.  For large problems,
the complexity of the matrix $B_k$ gets out of hand as one proceeds,
so we usually use tricks like {\em limited memory Broyden}, keeping
only a few recent updates to the initial (often diagonal) Jacobian
approximation.

\end{document}
