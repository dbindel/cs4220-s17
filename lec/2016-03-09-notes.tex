\documentclass[12pt, leqno]{article} %% use to set typesize 
\usepackage{amsfonts}
\usepackage{amsmath}
\usepackage{amssymb}
\usepackage{fancyhdr}
\usepackage{hyperref}
\usepackage{tikz}
\usepackage{pgfplots}
\usepackage{listings}

\newcommand{\bbR}{\mathbb{R}}
\newcommand{\bbC}{\mathbb{C}}
\newcommand{\calV}{\mathcal{V}}
\newcommand{\calW}{\mathcal{W}}
\newcommand{\ddiag}{\operatorname{diag}}
\newcommand{\fl}{\operatorname{fl}}
\newcommand{\macheps}{\epsilon_{\mathrm{mach}}}
\newcommand{\matlab}{\textsc{Matlab}}

\newcommand{\hdr}[2]{
  \pagestyle{fancy}
  \lhead{Bindel, Spring 2016}
  \rhead{Numerical Analysis (CS 4220)}
  \fancyfoot{}
  \begin{center}
    {\large{\bf Notes for #1}}
  \end{center}
  \lstset{language=matlab,columns=flexible}  
}


\newcommand{\uQ}{\underline{Q}}
\newcommand{\uR}{\underline{R}}

\begin{document}
\hdr{2016-03-09}

\section*{Orthogonal iteration to QR}

The {\em QR iteration} is the workhorse for solving the nonsymmetric
eigenvalue problem.  Unfortunately, while the iteration itself is
simple to write, the derivation sometimes appears to be a work of
black magic.  In fact, the QR iteration is essentially the subspace
iteration we have already seen, re-cast in a different form.

\begin{enumerate}
\item
  The orthogonal iteration $\uQ^{(k+1)} R^{(k)} = A \uQ^{(k)}$ is a
  generalization of the power method.  In fact, the first column of
  this iteration is {\em exactly} the power iteration.  In general,
  the first $p$ columns of $\uQ^{(k)}$ are converging to an orthonormal
  basis for a $p$-dimensional invariant subspace associated with the
  $p$ eigenvalues of $A$ with largest modulus (assuming that there
  aren't several eigenvalues with the same modulus to make this
  ambiguous).
\item
  If all the eigenvalues have different modulus, orthogonal iteration
  ultimately converges to the orthogonal factor in a Schur form
  \[
    AU = UT
  \]
  What about the $T$ factor?  Note that $T = U^* A U$, so a natural
  approximation to $T$ at step $k$ would be 
  \[
    A^{(k)} = (\uQ^{(k)})^* A \uQ^{(k)},
  \]
  and from the definition of the subspace iteration, we have
  \[
    A^{(k)} = (\uQ^{(k)})^* \uQ^{(k+1)} R^{(k)} = Q^{(k)} R^{(k)},
  \]
  where $Q^{(k)} \equiv (\uQ^{(k)})^* \uQ^{(k+1)}$ is unitary.
\item
  Note that
  \[
    A^{(k+1)} 
    = (\uQ^{(k+1)})^* A^{(k)} \uQ^{(k+1)} 
    = (Q^{(k)})^* A^{(k)} Q^{(k)}
    = R^{(k)} Q^{(k)}.
  \]
  Thus, we can go from $A^{(k)}$ to $A^{(k+1)}$ directly without
  the orthogonal factors from subspace iteration, simply by computing
  \begin{align*}
    A^{(k)} &= Q^{(k)} R^{(k)} \\
    A^{(k+1)} &= R^{(k)} Q^{(k)}.
  \end{align*}
  This is the QR iteration.
\end{enumerate}

\section*{Practical problems}

There are two major problems with the basic QR iteration:
\begin{enumerate}
\item
  Each step of the QR iteration requires a QR factorization, which is
  an $O(n^3)$ operation.  This is rather expensive, and even in the happy
  case where we might be able to get each eigenvalue with a constant
  number of steps, $O(n)$ total steps at a cost of $O(n^3)$ each gives
  us an $O(n^4)$ algorithm.  Given that everything else we have done
  so far costs only $O(n^3)$, an $O(n^4)$ cost for eigenvalue computation
  seems excessive.
\item
  Like the power iteration upon which it is based, the basic iteration
  converges linearly, and the rate of convergence is related to the
  ratios of the moduli of eigenvalues.  Convergence is slow when there
  are eigenvalues of nearly the same modulus, and nonexistent when
  there are eigenvalues with the same modulus.
\end{enumerate}
We now turn to each of these in turn.

\section*{Hessenberg matrices and $O(n^2)$ QR steps}

A matrix $H$ is {\em upper Hessenberg} if it has nonzeros only in the
upper triangle and the first subdiagonal.  For example, the nonzero
structure of a 5-by-5 Hessenberg matrix is
\[
  \begin{bmatrix}
    \times & \times & \times & \times & \times \\
    \times & \times & \times & \times & \times \\
           & \times & \times & \times & \times \\
           &        & \times & \times & \times \\
           &        &        & \times & \times 
  \end{bmatrix}.
\]
For any square matrix $A$, we can find a unitarily similar Hessenberg
matrix $H = Q^* A Q$ in $O(n^3)$ time (a topic for next time).
Because $H$ is similar to $A$, they have the same eigenvalues; but
as it turns out, the special structure of the Hessenberg matrix
makes it possible to run QR in $O(n^2)$ per iteration.

The special structure of the Hessenberg matrix makes the Householder
QR routine very economical.  The Householder reflection computed in
order to introduce a zero in the $(j+1,j)$ entry needs only to operate
on rows $j$ and $j+1$.  Therefore, we have
\[
  Q^* H = W_{n-1} W_{n-2} \ldots W_1 H = R,
\]
where $W_{j}$ is a Householder reflection that operates only on rows
$j$ and $j+1$.  Computing $R$ costs $O(n^2)$ time, since each $W_j$
only affects two rows ($O(n)$ data).  Now, note that
\[
  R Q = R (W_1 W_2 \ldots W_{n-1});
\]
that is, $RQ$ is computed by an operation that first mixes the first
two columns, then the second two columns, and so on.  The only subdiagonal
entries that can be introduced in this process lie on the first subdiagonal,
and so $RQ$ is again a Hessenberg matrix.  Therefore, one step of QR iteration
on a Hessenberg matrix results in another Hessenberg matrix, and a Hessenberg
QR iteration step can be performed in $O(n^2)$ time.

As it happens, the Hessenberg QR step can be written in terms of {\em
  bulge chasing}.  The picture (in the 5-by-5 case) is as follows.  We
start with the original Hessemberg matrix, and first apply an
orthogonal transformation (the first in a QR factorization) to the
first two rows and then symmetrically to the first two columns.  This
introduces a nonzero (a ``bulge'') in the $(3,1)$ position.  Marking
the elements modified in the first step with a star, we have:
\[
  \begin{bmatrix}
    * & * & * & * & * \\
    * & * & * & * & * \\
    * & * & \times & \times & \times \\
           &        & \times & \times & \times \\
           &        &        & \times & \times 
  \end{bmatrix}.
\]
From here, our goal is to ``chase'' the bulge out of the Hessenberg
structure.  We start by applying an orthogonal transformation to rows
2 and 3 to remove the $(3,1)$ element; applying the same
transformation to columns 2 and 3 introduces a bulge element in the
$(4,2)$ position; marking the newly modified elements with stars,
we have the new structure
\[
  \begin{bmatrix}
    \times & * & * & \times & \times \\
    * & * & * & * & * \\
    0 & * & * & * & * \\
           & * & * & \times & \times \\
           &        &        & \times & \times 
  \end{bmatrix}.
\]
Continuing for two more step in a similar fashion, we have a nonzero
in the $(5,3)$ position, and then can restore the structure the rest
of the way to a Hessenberg form.

\section*{Inverse iteration and the QR method}

When we discussed the power method, we found that we could improve
convergence by a spectral transformation that mapped the eigenvalue we
wanted to something with large magnitude (preferably much larger than
the other eigenvalues).  This was the {\em shift-invert} strategy.
We already know there is a connection leading from the power method
to orthogonal iteration to the QR method, which we can summarize with
a small number of formulas.  Let us see if we can follow the same
path to uncover a connection from inverse iteration (the power method
with $A^{-1}$, a special case of shift-invert in which the shift is zero) to QR.
If we call the orthogonal factors
in orthogonal iteration  $\uQ^{(k)}$ ($\uQ^{(0)} = I$) and the iterates 
in QR iteration $A^{(k)}$, we have
\begin{align}
  A^{k}   &= \uQ^{(k)} \uR^{(k)} \label{orth-iter-rel} \\
  A^{(k)} &= (\uQ^{(k)})^* A (\uQ^{(k)}).
\end{align}
In particular, note that because $R^{(k)}$ are upper triangular,
\[
  A^{k} e_1 = (\uQ^{(k)} e_1) r^{(k)}_{11};
\]
that is, the first column of $\uQ^{(k)}$ corresponds to the $k$th
step of power iteration starting at $e_1$.  What happens when we
consider negative powers of $A$?  Inverting (\ref{orth-iter-rel}),
we find
\[
  A^{-k} = (\uR^{(k)})^{-1} (\uQ^{(k)})^*
\]
The matrix $\tilde{R}^{(k)} = (\uR^{(k)})^{-1}$ is again upper triangular;
and if we look carefully, we can see in this fact another power iteration:
\[
  e_n^* A^{-k} = e_n^* \tilde{R}^{(k)} (\uQ^{(k)})^* 
              = \tilde{r}^{(k)}_{nn} (\uQ^{(k)} e_n)^*.
\]
That is, the last column of $\uQ^{(k)}$ corresponds to a power iteration
converging to a {\em row} eigenvector of $A^{-1}$.

\section*{Shifting gears}

The connection from inverse iteration to orthogonal iteration (and
thus to QR iteration) gives us a way to incorporate the shift-invert
strategy into QR iteration: simply run QR on the matrix $A-\sigma I$,
and the $(n,n)$ entry of the iterates (which corresponds to a Rayleigh
quotient with an increasingly-good approximate row eigenvector) should
start to converge to $\lambda - \sigma$, where $\lambda$ is the
eigenvalue nearest $\sigma$.  Put differently, we can run the
iteration:
\begin{align*}
  Q^{(k)} R^{(k)} &= A^{(k-1)} - \sigma I \\
  A^{(k)} &= R^{(k)} Q^{(k)} + \sigma I.
\end{align*}
If we choose a good shift, then the lower right corner entry of
$A^{(k)}$ should converge to the eigenvalue closest to $\sigma$ in
fairly short order, and the rest of the elements in the last row
should converge to zero.

The shift-invert power iteration converges fastest when we choose a
shift that is close to the eigenvalue that we want.  We can do even
better if we choose a shift {\em adaptively}, which was the basis for
running Rayleigh quotient iteration.  The same idea is the basis
for the {\em shifted QR iteration}:
\begin{align}
  Q^{(k)} R^{(k)} &= A^{(k-1)} - \sigma_k I \label{sqr-1} \\
  A^{(k)} &= R^{(k)} Q^{(k)} + \sigma_k I. \label{sqr-2}
\end{align}
This iteration is equivalent to computing
\begin{align*}
  \uQ^{(k)} \uR^{(k)} &= \prod_{j=1}^n (A-\sigma_j I) \\
  A^{(k)} &= (\uQ^{(k)})^* A (\uQ^{(k)}) \\
  \uQ^{(k)} &= Q^{(k)} Q^{(k-1)} \ldots Q^{(1)}.
\end{align*}
When we have transformed in advance to Hessenberg form, shifting can
be naturally incorporated into the ``bulge-chasing'' picture.  The
only thing that changes is the choice of the first orthogonal
transformation, which is chosen based on looking at the first step of
QR factorization on the shifted matrix rather than the original matrix.

What should we use for the shift parameters $\sigma_k$?  A natural
choice is to use $\sigma_k = e_n^* A^{(k-1)} e_n$, which is the same
as $\sigma_k = (\uQ^{(k)} e_n)^* A (\uQ^{(k)} e_n)$, the Rayleigh quotient
based on the last column of $\uQ^{(k)}$.  This simple shifted QR iteration
is equivalent to running Rayleigh iteration starting from an initial vector
of $e_n$, which we noted before is locally quadratically convergent.
This strategy (Wilkinson shifts) is close to what is done in practice,
though we usually do something special when the matrix is real in
order to apply pairs of complex shifts together (the ``double shift''
strategy), and an occasional ``exceptional'' shift is needed to keep
the iteration from getting stuck at times.

\section*{Modern QR and beyond}

Modern QR iteration involves several tricks to make things run fast on
large matrices.  {\em Multi-shift} techniques get several shifts at a
time to run overlapping QR iterations; {\em aggressive early
  deflation} techniques similarly take advantage of the fact that
there may be several eigenpairs that converge nearly simultaneously.
A practical implementation that takes into account all these tricks is
a bit more complex than the basic QR iteration, enough that it usually
makes sense to rely on the codes in LAPACK (or in MATLAB or SciPy,
which call LAPACK).

So far, we have considered the case of {\em general nonsymmetric}
matrices.  In the {\em symmetric} case, the Hessenberg reduction takes
one all the way to a tridiagonal, and it is possible to do one QR step
in $O(n)$ time.  There are more efficient methods than QR for
computing all the eigenvalues of a tridiagonal; but even for QR, the
initial tridiagonal reduction (which costs $O(n^3)$) dominates the
cost of the later work.

\end{document}
