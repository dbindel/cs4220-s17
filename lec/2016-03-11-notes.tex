\documentclass[12pt, leqno]{article} %% use to set typesize 
\usepackage{amsfonts}
\usepackage{amsmath}
\usepackage{amssymb}
\usepackage{fancyhdr}
\usepackage{hyperref}
\usepackage{tikz}
\usepackage{pgfplots}
\usepackage{listings}

\newcommand{\bbR}{\mathbb{R}}
\newcommand{\bbC}{\mathbb{C}}
\newcommand{\calV}{\mathcal{V}}
\newcommand{\calW}{\mathcal{W}}
\newcommand{\ddiag}{\operatorname{diag}}
\newcommand{\fl}{\operatorname{fl}}
\newcommand{\macheps}{\epsilon_{\mathrm{mach}}}
\newcommand{\matlab}{\textsc{Matlab}}

\newcommand{\hdr}[2]{
  \pagestyle{fancy}
  \lhead{Bindel, Spring 2016}
  \rhead{Numerical Analysis (CS 4220)}
  \fancyfoot{}
  \begin{center}
    {\large{\bf Notes for #1}}
  \end{center}
  \lstset{language=matlab,columns=flexible}  
}


\newcommand{\uQ}{\underline{Q}}
\newcommand{\uR}{\underline{R}}

\begin{document}
\hdr{2016-03-11}

\section*{Woah, We're Halfway There}

Last time, we showed that the QR iteration maps upper Hessenberg
matrices to upper Hessenberg matrices, and this fact allows us to do
one QR sweep in $O(n^2)$ time.  We did not, however, discuss how to
get to upper Hessenberg form.  This lecture is devoted to that
reduction, and to other ``halfway there'' forms:
\begin{align*}
  A &= Q H Q^T \quad \mbox{where $Q$ is orthogonal and $H$ upper
    Hessenberg} \\
  A &= Q T Q^T \quad \mbox{where $Q$ is orthogonal, $T$ symmetric
    tridiagonal, and $A$ symmetric} \\
  A &= U B V^T \quad \mbox{where $U$ and $V$ are orthogonal and $B$ is
    upper bidiagonal}
\end{align*}
reduction of
Unlike the SVD or various eigendecompositions, these forms can be
computed directly.  And in many cases, as we will discuss, they are
just as useful!

\section*{Hessenberg via Householder}

Let's start with an over-ambitious goal\footnote{Ah, but a man's reach
  should exceed his grasp...}: can we directly compute the Schur form
by applying Householder transformations on the left and right of a
matrix?  We already secretly know that the answer is no, but trying
and failing will set us up neatly for computing a Hessenberg form ---
and for understanding why the Hessenberg form is in fact a natural
target.

How might we go about trying to compute a Schur form directly?  A
natural first step is to apply an orthogonal transformation from the
left in order to introduce zeros in the first column, and then apply
the same transformation on the right.  Let's try this for the $4
\times 4$ matrix case (we'll mark with a star each nonzero element
that is updated by the previous transformation):
\[
\begin{bmatrix}
  \times & \times & \times & \times \\
  \times & \times & \times & \times \\
  \times & \times & \times & \times \\
  \times & \times & \times & \times
\end{bmatrix} \mapsto
\begin{bmatrix}
  * & * & * & * \\
  0 & * & * & * \\
  0 & * & * & * \\
  0 & * & * & *
\end{bmatrix} \mapsto
\begin{bmatrix}
  * & * & * & * \\
  * & * & * & * \\
  * & * & * & * \\
  * & * & * & *
\end{bmatrix}.
\]
Oops!  Applying the transformation from the right scrambles the
columns, and undoes the zeros we started just introduced.  The
problem is that we got too greedy.  What if instead of trying to
zero out everything below the diagonal, we instead zero out everything
below the first subdiagonal?  Then we affect rows 2 through $n$,
and a subsequent transform affecting columns 2 through $n$ does not
kill the zeros we introduced:
\[
\begin{bmatrix}
  \times & \times & \times & \times \\
  \times & \times & \times & \times \\
  \times & \times & \times & \times \\
  \times & \times & \times & \times
\end{bmatrix} \mapsto
\begin{bmatrix}
  \times & \times & \times & \times \\
  * & * & * & * \\
  0 & * & * & * \\
  0 & * & * & *
\end{bmatrix} \mapsto
\begin{bmatrix}
  \times & * & * & * \\
  \times & * & * & * \\
  0 & * & * & * \\
  0 & * & * & *
\end{bmatrix}.
\]
A second step to introduce zeros in the second column finishes the
reduction to Hessenberg form:
\[
\begin{bmatrix}
  \times & \times & \times & \times \\
  \times & \times & \times & \times \\
  \times & \times & \times & \times \\
  \times & \times & \times & \times
\end{bmatrix} \mapsto
\begin{bmatrix}
  \times & \times & \times & \times \\
  \times & \times & \times & \times \\
  0 & * & * & * \\
  0 & 0 & * & *
\end{bmatrix} \mapsto
\begin{bmatrix}
  \times & \times & * & * \\
  \times & \times & * & * \\
  0 & \times & * & * \\
  0 & 0 & * & *
\end{bmatrix}.
\]
Let's make this concrete:
\lstinputlisting{2016-03-11-codes/hess_house.m}

\section*{Terrific Tridiagonals and Busy Bidiagonals}

When $A$ is a symmetric matrix, the corresponding Hessenberg form
is symmetric as well.  This means that all the entries
below the first subdiagonal are zero (by upper Hessenberg structure),
and so are all the entries above the first superdiagonal (by
symmetry).

What about bidiagonalization?  In bidiagonalization, we again
interleave transformations on the left and right, but now we allow
ourselves to use different transformations.  The key is that we
want to keep introducing new zero elements through these
transformations without destroying zeros we already created.
In the $4 \times 4$ case, the first two steps look like
\[
\begin{bmatrix}
  \times & \times & \times & \times \\
  \times & \times & \times & \times \\
  \times & \times & \times & \times \\
  \times & \times & \times & \times
\end{bmatrix} \mapsto
\begin{bmatrix}
  * & * & * & * \\
  0 & * & * & * \\
  0 & * & * & * \\
  0 & * & * & *
\end{bmatrix} \mapsto
\begin{bmatrix}
  \times & * & 0 & 0 \\
  0 & * & * & * \\
  0 & * & * & * \\
  0 & * & * & *
\end{bmatrix} 
\]

The code is
\lstinputlisting{2016-03-11-codes/bidiag_house.m}

\section*{Using the Factorizations}

Let's look concretely at two cases where these factorizations are
useful.  The first is in computing {\em transfer functions}
that occur in control theory, where we have the form
\[
  T(s) = c^T (sI-A)^{-1} b + d
\]
for various values of $s$.  Naively, it looks like we might have
to spend $O(n^3)$ per value of $s$ where we want the transfer
function; but we can instead use the Hessenberg reduction
$A = Q H Q^T$ to get
\[
  T(s) = (c^T Q) (sI-H)^{-1} (Q^T b) + d
\]
Solving a Hessenberg linear system like $(sI-H)$ can be done in
$O(n^2)$ time rather than $O(n^3)$ time (why?); in fact, this is one
of the structures that MATLAB's sparse solver checks for.

A second application again deals with parameter-dependent
systems, but in a different setting.  Suppose $A = UBV^T$ is a
bidiagonal reduction of $A$, and we are interested in computing
the Tikhonov-regularized solution for different values of the
regularization parameter.  How can we do this efficiently?
Note that
\[
\left\|
\begin{bmatrix} A \\ \lambda I \end{bmatrix} x -
\begin{bmatrix} b \\ 0 \end{bmatrix}
\right\| =
\left\|
\begin{bmatrix}
  B \\
  \lambda I
\end{bmatrix}
(V^T x) -
\begin{bmatrix}
  U^T b \\
  0
\end{bmatrix}
\right\|
\]
The latter equations can be reduced back to bidiagonal form in $O(n)$
time (left as an exercise for the student!).  Therefore, after the
initial bidiagonal reduction, we can solve the least squares
problem for each new value of the regularization parameter $\lambda$
in $O(n)$ additional work.

\section*{Enter Arnoldi}

When we discussed QR factorization, we started with Gram-Schmidt and
then moved to the Householder-based method.  In talking about
reduction to Hessenberg form, we started with the Householder approach
--- but there is something akin to Gram-Schmidt as well.  We can read
the basic idea by looking at the columns of the Hessenberg matrix
equation $A Q = Q H$:
\[
  A q_{j} = \sum_{k=1}^j q_j h_{jk} + h_{j,j+1} q_{j+1}.
\]
Rearranging, we have
\[
  q_{j+1} = \frac{1}{h_{j,j+1}} \left( A q_{j} - \sum_{k=1}^j q_k
  h_{kj} \right)
\]
where $h_{kj} = q_k^T A q_j$.  That is, the coefficients $H$ can be
exactly derived from using Gram-Schmidt orthgonalization to
orthonormalize $A q_j$ against $\{q_1, \ldots, q_j\}$ for each
successive $j$!

The beauty about using the Arnoldi procedure to reduce a matrix to
upper Hessenberg form is two fold: we can readily apply
the method to sparse matrices; and we can {\em stop early!}.
This latter fact is true of the Householder-based methods as well.
But in the case of the Arnoldi procedure, we will end up making
use of partial Hessenberg reduction to turn Arnoldi into a method of
approximating linear system solutions (GMRES) and a method of
approximating eigenpairs for large systems (the Arnoldi method).

When $A$ is symmetric, the Arnoldi procedure becomes the {\em Lanczos}
procedure.  Note that in this case, we {\em tridiagonalize} the
original matrix, which means at each step we need (in exact
arithmetic) to orthogonalize each successive $Aq_j$ against only two
other vectors.  This rather remarkable fact allows us to parley the
Lanczos iteration into an eigenvalue iteration as well as two truly
remarkable iterations for solving linear systems: MINRES and the
famous method of conjugate gradients (CG).  We will return to these
methods in about a week.

\end{document}
