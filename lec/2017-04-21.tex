\documentclass[12pt, leqno]{article} %% use to set typesize 
\usepackage{amsfonts}
\usepackage{amsmath}
\usepackage{amssymb}
\usepackage{fancyhdr}
\usepackage{hyperref}
\usepackage{tikz}
\usepackage{pgfplots}
\usepackage{listings}

\newcommand{\bbR}{\mathbb{R}}
\newcommand{\bbC}{\mathbb{C}}
\newcommand{\calV}{\mathcal{V}}
\newcommand{\calW}{\mathcal{W}}
\newcommand{\ddiag}{\operatorname{diag}}
\newcommand{\fl}{\operatorname{fl}}
\newcommand{\macheps}{\epsilon_{\mathrm{mach}}}
\newcommand{\matlab}{\textsc{Matlab}}

\newcommand{\hdr}[2]{
  \pagestyle{fancy}
  \lhead{Bindel, Spring 2016}
  \rhead{Numerical Analysis (CS 4220)}
  \fancyfoot{}
  \begin{center}
    {\large{\bf Notes for #1}}
  \end{center}
  \lstset{language=matlab,columns=flexible}  
}


\begin{document}
\hdr{2017-04-21}

\section*{Pacing the Path}

So far, we have focused on
nonlinear equations ($f : \bbR^n \rightarrow \bbR^n$),
and to a lesser extent on
optimization problems ($\phi : \bbR^n \rightarrow \bbR$).
Often, though, nonlinear equations and optimization problems
depend on some extra parameter.  For example:
\begin{itemize}
\item
  In fitting problems, we care about the solution as a function
  of a regularization parameter.
\item
  In population biology, we care about equilibrium population levels
  as a function of various parameters: birth rates, death rates,
  initial population sizes, etc.
\item
  In mechanics, we care about the deformation of a structure
  as a function of load.
\item
  In chemical kinetics, we care about equilibrium chemical
  concentrations as a function of temperature.
\item
  In engineering problems involving a tradeoff between two parameters
  (e.g.~mass and stiffness), we care about the optimal setting of one
  parameter given a fixed value of the other.
\item
  In stochastic problems, we may care about the behavior as a function of
  the variance of the noise term, or perhaps as a function of an
  autocorrelation time.
\end{itemize}
For these types of problems, {\em continuation} strategies are often
a good choice.  The basic picture in a continuation strategy for
solutions of an equation $f(x(s),s) = 0$ where
$f : \bbR^n \times \bbR \rightarrow \bbR^n$
starting from some easily computed solution $x(s_0)$ is:
\begin{itemize}
\item
  Given $x(s_j)$, choose a new $s' = s_j + \Delta s$.
\item
  {\em Predict} $x(s')$ based on the behavior at $s$.  Two common
  predictors are
  \begin{itemize}
  \item {\em Trivial}: Guess $x(s') \approx x(s_j)$.
  \item {\em Euler:} Guess
    $x(s') \approx x(s_j) - \frac{\partial f}{\partial s}(x_j,s_j)^{-1} \frac{\partial f}{\partial s}(x_j,s_j)$.
  \end{itemize}
\item
  {\em Correct} by taking a few steps of Newton iteration.
\item
  Either {\em accept} $s_{j+1} = s'$ and a corresponding $x(s_j)$ if
  the Newton iteration converged, or try again with a smaller $\Delta
  s$.  If the Newton iteration converges very quickly, we may
  increase $\Delta s$.
\end{itemize}

Continuation is also natural if we really do care about a problem with no free
parameters, but we lack a good initial guess with which to start an
iterative method to solve the problem.  In that case, a reasonable
strategy is often to {\em introduce} a parameter $s$ such that the
equations at $s = 0$ are easy and the equations at $s = 1$ are the
ones that we would like to solve.  Such a constructed path in problem
space is sometimes called a {\em homotopy}.  In many cases, one can
show that solutions are continuous (though not necessarily
differentiable) functions of the homotopy parameter, so that following
a homotopy path with sufficient care can provide {\em all} solutions
even for hard nonlinear problems.  For this reason, homotopy methods
are particularly effective for solving systems of polynomial
equations.  Another very popular family of homotopy methods are the
interior point methods for constrained optimization problems,
which we will touch on briefly next week.

\section*{Tough to Trace}

As a starting example, let's consider a variation on
the equation from one of our first nonlinear systems lectures,
a discretization of the thermal blowup equation
\[
  \frac{d^2 u}{dx^2} + \exp(\gamma u) = 0
\]
subject to $u(0) = u(1) = 0$.  We approximate the derivative
with a mesh to get a system of equations of the form
\[
  h^{-2} \left( u_{j-1}-2u_j+u_{j+1} \right) + \exp(\gamma u_j) = 0
\]
where $u_j$ is the approximate solution at a mesh point $x_j = jh$
with $h = 1/(N+1)$.  The boundary conditions are $u_0 = u_{N+1} = 0$,
and the difference equations govern the behavior on the interior.
Compared to the last time we saw this system, though, we have
introduced a new feature: the rate constant $\gamma$.

When $\gamma$ is equal to zero, the differential equation becomes
\[
  \frac{d^2 u}{dx^2} + 1 = 0,
\]
which has the solution (subject to boundary conditions)
\[
  u(x; 0) = \frac{1}{2} x(1-x).
\]
For larger values of $\gamma$, things become more interesting.
Based on physical reasoning, we expect the solutions to get more
unstable (and harder) as $\gamma$ grows.  We therefore consider
a strategy in which we incrementally increase $\gamma$, at each
point using a trivial predictor (the solution for the previous
$\gamma$) as an initial guess for a Newton iteration.  If the
Newton iteration does not converge in a few steps, we try again
with a smaller step, stopping once the step size has become too small.

\begin{figure}
\begin{tikzpicture}
  \begin{axis}[width=0.45\textwidth,height=5cm,xlabel={$\gamma$},ylabel={$v(0.5)$},
      scaled y ticks=base 10:2]
    \addplot table {data/vcenter.txt};
  \end{axis}
\end{tikzpicture}
\hspace{0.08\textwidth}
\begin{tikzpicture}
  \begin{semilogyaxis}[width=0.45\textwidth,height=5cm,xlabel={$\gamma$},ylabel={$\Delta \gamma$}]
    \addplot table {data/dgammas.txt};
  \end{semilogyaxis}
\end{tikzpicture}
\caption{Center solution and step size versus $\gamma$}
\label{fig:vsgamma}
\end{figure}

The behavior of the algorithm as a function of $\gamma$ is shown
in Figure~\ref{fig:vsgamma}.  As we get just past $\gamma = 3.5$,
the solution becomes more and more sensitive to small changes in
$\gamma$, and we have to take shorter steps in order to get
convergence.  This is reflective of an interesting physical
phenomenon known (a bifurcation).  Mathematically, what we see
is the effect of the Jacobian becoming closer and closer to singular
at the solution.

\section*{Picking Parameters}

\begin{figure}
\begin{tikzpicture}
  \begin{axis}[width=\textwidth,height=5cm,xlabel={$\gamma$},ylabel={$v(0.5)$},
      scaled y ticks=base 10:2]
    \addplot table {data/vcenterm.txt};
  \end{axis}
\end{tikzpicture}
\caption{Center solution versus $\gamma$ traced by continuation in the
  midpoint value.}
\label{fig:vsgammam}
\end{figure}

In the previous section, we saw that continuation allowed us to march
$\gamma$ right up to some critical parameter, but not beyond.
We can get a clearer picture of what is going on --- and better solver
stability --- if we look at the same problem as a function of a
{\em different parameter}.  In particular, let us consider
controlling the midpoint value $\mu$, and letting both $v$
and $\gamma$ be implicit functions of the midpoint value.  That is,
we have the equations
\[
F(v,\gamma; \mu) =
\begin{bmatrix}
  -h^{-2} T_N v + \exp(\gamma v) \\
  e_{\mathrm{mid}}^T v - \mu
\end{bmatrix} = 0
\]
with the Jacobian matrix (with respect to $v$ and $\gamma$)
\[
\frac{\partial F}{\partial (v,\gamma)} =
\begin{bmatrix}
  -h^2 T_n + \gamma \operatorname{diag}(\exp(\gamma v)) &
  v \odot \exp(\gamma v) \\
  e_{\mathrm{mid}}^T & 0
\end{bmatrix}
\]
where we use $a \odot b$ to denote elementwise multiplication.
If we use the same continuation process with a trivial predictor
to trace out the behavior of the midpoint as a function of $\gamma$,
we obtain Figure~\ref{fig:vsgammam}.  This picture makes the behavior
of the solution close to $\gamma = 3.5$ a little more clear.  The
phenomenon shown is called a {\em fold bifurcation}.  Physically,
we have that for $\gamma \lesssim 3.5$, there are two distinct
solutions (one stable and one unstable); as $\gamma$ increases, these
two solutions approach each other until at some critical $\gamma$
value they ``meet.''  Beyond the critical value, there is no solution
to the equations.

\section*{Pseudoarclength Ideas}

What if we think we might run into a fold bifurcation, but do not
know a good alternate parameter for continuation?  A natural idea
is to parameterize the solution curve (e.g.~$(v(\gamma),\gamma)$)
in terms of an {\em arclength} parameter.  In practice, we do not
care too much about exactly controlling arclength; it is just a
mechanism to avoid picking parameters.  Therefore, we pursue
{\em pseudo-arclength} strategies as an alternative.

For the simplest pseudo-arclength continuation strategy, consider
a function $F : \bbR^{n+1} \rightarrow \bbR^n$.  Assuming the
Jacobian has maximal rank, we expect there to be a solution curve
$x : \bbR \rightarrow \bbR^n$ such that $F(x(s)) = 0$.
The null vector of the Jacobian $F'$ is tangent to $x$, and so
we can use this to predict a new point.  The basic procedure
to get a new point on the curve starting from $x^j$ is then:
\begin{itemize}
\item
  Consider the Jacobian $F'(x^j) \in \bbR^{n \times (n+1)}$ and
  compute a null vector $v$ (a simple approach is to compute a
  QR factorization).  Choose a tangent vector $t^j \propto v$;
  usually we normalize so that $t^{j-1} \cdot t^j > 0$.
\item
  Move a short distance along the tangent direction (Euler predictor),
  or otherwise predict a new point.
\item
  Correct back to the curve by the iteration
  \[
    y^{k+1} = y^k - F'(y^k)^\dagger F(y^k)
  \]
  where $F'(y^k)^\dagger \in \bbR^{(n+1) \times n}$ is the
  pseudoinverse of the Jacobian.  This is equivalent to solving
  the problem
  \[
    \mbox{minimize } \|p^k\|^2 \mbox{ s.t. } F'(y^k) p^k = -F(y^k).
  \]
  The steps of this underdetermined system should quickly take us back to
  a new point on the curve.
\item
  If the iteration curves and the new point is OK, accept the point
  and move on.  Otherwise, reject the point and try again with a
  shorter step in the tangent direction.
\end{itemize}

\section*{And Points Beyond}

There is a large and fascinating literature on numerical continuation
methods and on the numerical analysis of implicitly defined functions.
Beyond the predictor-corrector methods that we have described, there
are various other methods that address similar problems: piecewise
linear (simplex) continuation, pseudo-transient continuation, and so
forth.  We can combine continuation ideas with all the other
ideas that we have described in the course; for example, one can do
clever things with Broyden updates as one walks along the curve.
We can also apply step control techniques that some of you may have
learned in a class like CS 4210 in the context of methods for solving
ordinary differential equations.

A little knowledge of continuation methods can take you a long way,
but if you would like to know more, I recommend
{\em Introduction to Numerical Continuation Methods} by
Allgower and Georg.

\section*{Appendix: Codes}

\subsection*{Continuation in $\gamma$}

\lstinputlisting{code/blowupc.m}

\subsection*{Continuation in $\mu$}

\lstinputlisting{code/blowupmid.m}


\end{document}
