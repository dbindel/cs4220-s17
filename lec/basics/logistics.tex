\section{Logistics}

We will go through the syllabus in detail, but at a high level you
should plan on six homeworks (individual) and three projects (in
pairs), a midterm, and a final.  I will also ask you for feedback at
the middle and end of the semester, and this counts for credit.  I
will give ``problems of the day'' to help study, but we will not use
these directly to grade you.

Homework and projects are due via CMS by midnight on Fridays; we allow
some ``slip days'' so that you can work on an assignment through the
weekend if needed.  We will also drop the lowest of the HW grades, in
case there is a particularly hectic week.  We have office hours
scheduled before class on Wednesday; the remaining office hours are TBD
depending, but we will announce them soon.  You can also request office
hours by appointment.

\subsection{Infrastructure}

Class notes and assignments, as well as class announcements, will be
posted on the course home page.  For submissions, solutions, and
grades, we will use the CS Course Management System (CMS) software.
For class discussion, we will use Piazza.  There are links from each
of these pages to the others; I recommend you use the class web page
as your starting point.

We will use MATLAB and Julia in our notes.  You may use either one
for your homework; we will distinguish based on the extension.

The course web page is maintained from a repository on GitHub.
I encourage you to submit corrections or enhancements by pull
request!

\subsection{Background}

The formal prerequisites for the class are linear algebra at the level
of Math 2210 or 2940 or equivalent and a CS 1 course in any language.
We also recommend one additional math course at the 3000 level or
above; this is essentially a proxy for ``sufficient mathematical
maturity.''

In practice: I will assume you know some multivariable calculus
and linear algebra, and that your CS background includes not only
basic programming but also some associated mathematical concepts
(e.g.~order notation and a little graph theory).  If you feel your
background is weak in these areas, please talk to us.

Some of you may want to review your linear algebra basics in particular.
At Cornell, our undergraduate linear algebra course uses the text
by Lay~\cite{Lay:2016:Linear}; the texts by Strang~\cite{Strang:2006:Linear,Strang:2009:Introduction} are a nice
alternative.  Strang's {\em Introduction to Linear Algebra}~\cite{Strang:2009:Introduction} is the textbook for the MIT
linear algebra course that is the basis for his enormously popular
video lectures, available on MIT's OpenCourseWare site; if you prefer
lecture to reading, Strang is known as an excellent lecturer.
