\documentclass[12pt, leqno]{article} %% use to set typesize 
\usepackage{amsfonts}
\usepackage{amsmath}
\usepackage{amssymb}
\usepackage{fancyhdr}
\usepackage{hyperref}
\usepackage{tikz}
\usepackage{pgfplots}
\usepackage{listings}

\newcommand{\bbR}{\mathbb{R}}
\newcommand{\bbC}{\mathbb{C}}
\newcommand{\calV}{\mathcal{V}}
\newcommand{\calW}{\mathcal{W}}
\newcommand{\ddiag}{\operatorname{diag}}
\newcommand{\fl}{\operatorname{fl}}
\newcommand{\macheps}{\epsilon_{\mathrm{mach}}}
\newcommand{\matlab}{\textsc{Matlab}}

\newcommand{\hdr}[2]{
  \pagestyle{fancy}
  \lhead{Bindel, Spring 2016}
  \rhead{Numerical Analysis (CS 4220)}
  \fancyfoot{}
  \begin{center}
    {\large{\bf Notes for #1}}
  \end{center}
  \lstset{language=matlab,columns=flexible}  
}


\begin{document}
\hdr{2017-05-05}

\section{Lay of the Land}

In the landscape of continuous optimization problems, there are
two axes in which things can be hard.  In the first axis,
we have the global structure of the problem:
\begin{itemize}
\item
  Easiest: Quadratic functions.
\item
  Harder: General convex functions.
\item
  Harder: Nonconvex functions with special structure.
\item
  Harder: Nonconvex functions with some smoothness.
\item
  Hardest: Nonconvex functions with wild variations.
\end{itemize}
Along the second axis, we have the local structure that is
readily available:
\begin{itemize}
\item
  Easy: Hessians and gradients are available.
\item
  Harder: Gradients are available, but Hessians are not (or are too
  costly to use).
\item
  Hardest: Gradients are not available; only function evaluations.
\end{itemize}

For the past two weeks, we have discussed the related problems of
optimization and nonlinear equation solving.  Our standard method
(Newton iteration) applies to both problems.  In the categorization
above, this is ``easy'' in terms of assuming we know lots of local
structure: gradients and Hessians.  Newton converges quickly
from good guesses, and can be effectively globalized, making it
useful not only for convex problems but also for harder non-convex
problems (given a good initial guess).  But Newton uses
many derivatives, and factorizations that may be expensive.  We
have discussed a few methods (e.g.~Broyden) that require fewer
derivatives, but we can go further.

Today we discuss optimization in one of the hard cases.  For this
class, we will not deal with the case of problems with very hard
global structure, other than to say that this is a land where
heuristic methods (simulated annealing, genetic algorithms, and
company) may make sense.  But there are some useful methods that are
available for problems where the global structure is not so hard as to
demand heuristics, but the problems are hard in that they are ``black
box'' we are limited in what we can compute to looking at function
evaluations.

Before describing some methods, I make a plea that you consider
these only after having thoughtfully weighed the pros and cons of
gradient-based methods.  If the calculus involved in computing
the derivatives is too painful, consider a computer algebra system,
or look into a tool for automatic differentiation of computer
programs.  Alternately, consider whether there are numerical estimates
of the gradient (via finite differences) that can be computed more
quickly than one might expect by taking advantage of the structure
of how the function depends on variables.  But if you really have to
work with a black box code, or if the pain of computing derivatives
(even with a tool) is too great, a gradient-free approach may be for you.

\section{Model-based methods}

The idea behind Newton's method is to successively minimize a
quadratic {\em model} of the function behavior based on a second-order
Taylor expansion about the most recent guess, i.e. $x^{k+1} = x^k + p$
where
\[
  \operatorname{argmin}_{p} \phi(x) + \phi'(x) p + \frac{1}{2} p^T H(x) p.
\]
In some Newton-like methods, we use a more approximate model, usually
replacing the Hessian with something simpler to compute and factor.
In simple gradient-descent methods, we might fall all the way back to
a linear model, though in that case we cannot minimize the model
globally -- we need some other way of controlling step lengths.
We can also explicitly incorporate our understanding of the quality of
the model by specifying a constraint that keeps us from moving outside
a ``trust region'' where we trust the model to be useful.

In derivative-free methods, we will keep the basic ``minimize the
model'' approach, but we will use models based on interpolation
(or regression) in place of the Taylor expansions of the Newton
approach.  There are several variants.

\subsection{Finite difference derivatives}

Perhaps the simplest gradient-free approach (though not necessarily
the most efficient) takes some existing gradient-based approach
and replaces gradients with finite difference approximations.
There are a two difficulties with this approach:
\begin{itemize}
  \item If $\phi : \bbR^n \rightarrow \bbR$, then computing the
    $\nabla \phi(x)$ by finite differences involves at least $n+1$
    function evaluations.  Thus the typical cost per step ends up
    being $n+1$ function evaluations (or more), where methods that are
    more explicitly designed to live off samples might only use a
    single function evaluation per step.
  \item The finite difference approximations depends on a
    step size $h$, and their accuracy is a complex function of $h$.
    For $h$ too small, the error is dominated by cancellation,
    revealing roundoff error in the numerical function evaluations.
    For $h$ large, the error depends on both the step size and the
    local smoothness of the function.
\end{itemize}

\subsection{Linear models}

A method based on finite difference approximations of gradients might
use $n+1$ function evaluations per step: one to compute a value at
some new point, and $n$ more in a local neighborhood to compute values
to estimate derivatives.  An alternative is to come up with an
approximate linear model for the function using $n+1$ function
evaluations that may include some ``far away'' function evaluations
from previous steps.

We insist that the $n+1$ evaluations form a simplex with nonzero
volume; that is, to compute from evaluations at points $x_0, \ldots,
x_n$, we want $\{ x_j-x_0 \}_{j=1}^n$ to be linearly independent
vectors.  In that case, we can build a model $x \mapsto b^T x + c$
where $b \in \bbR^n$ and $c \in \bbR$ are chosen so that the model
interpolates the function values.  Then, based on this model,
we choose a new point.

There are many methods that implicitly use linear approximations
based on interpolation over a simplex.  One that uses the concept
rather explicitly is Powell's COBYLA (Constrained Optimization BY
Linear Approximation), which combines a simplex-based linear
approximation with a trust region.

\subsection{Quadratic models}

One can build quadratic models of a function from only function
values, but to fit a quadratic model in $n$-dimensional space, we
usually need $(n+2)(n+1)/2$ function evaluations -- one for each of
the $n(n+1)/2$ distinct second partials, and $n+1$ for the linear
part.  Hence, purely function-based methods that use quadratic models
tend to be limited to low-dimensional spaces.  However, there are
exceptions.  The NEWUOA method (again by Powell) uses $2n+1$ samples
to build a quadratic model of the function with a diagonal matrix at
second order, and then updates that matrix on successive steps in a
Broyden-like way.

\subsection{Response surfaces}

Polynomial approximations are useful, but they are far from the only
methods for approximating objective functions in high-dimensional
spaces.  One popular approach\footnote{%
At least, it is popular that I've gotten pulled into working on it.
Your TA does, too!  See \url{https://github.com/dme65/pySOT}.  
} is to use {\em radial basis functions};
for example, we might write a model
\[
  s(x) = \sum_{j=1}^m c_j \phi(\|x-x_j\|)
\]
where the coefficients $c_j$ are chosen to satisfy $m$ interpolation
conditions at points $x_1, \ldots, x_m$.  Another option is to use
a Gaussian process model to predict how the objective function behaves
between objectives; this is used, for example, in an optimizer called
EGO.  There are a variety of other surfaces one might consider, though.

In addition to fitting a surface that interpolates known function
values, there are also methods that use {\em regression} to fit
some set of known function values in a least squares sense.
This is particularly useful when the function values have noise.

\section{Pattern search and simplex}

So far, the methods we have described are explicit in building a model
that approximates the function.  However, there are also methods that
use a systematic search procedure in which a model does not explicitly
appear.  These sometimes go under the heading of ``direct search''
methods.

\subsection{Nelder-Mead}

The Nelder-Mead algorithm is one of the most popular derivative-free
optimizers around.  For example, this is the default algorithm used
in MATLAB's {\tt fminsearch}.  As with methods like COBYLA, the
Nelder-Mead approach maintains a simplex of $n+1$ function evaluation
points that it updates at each step.  In Nelder-Mead,
one updates the simplex based on function values at the simplex
corners, the centroid, and one other point; or one contracts the simplex.

Visualizations of Nelder-Mead are often quite striking: the simplex
appears to crawl downhill like some sort of mathematical amoeba.
But there are examples of functions where Nelder-Mead is not
guaranteed to converge to a minimum at all.

\subsection{Hook-Jeeves and successors}

The basic idea of {\em pattern search} methods is to test points in a
pattern around the current ``best'' point.  For example, in the
Hook-Jeeves approach (one of the earliest pattern search methods),
one would at each move evaluate $\phi(x^{(k)} \pm \Delta e_j)$ for each
of the $n$ coordinate directions $e_j$.  If one of the new points is
better than $x^{(k)}$, it becomes $x^{(k+1)}$ (and we may increase
$\Delta$ if we already took a step in this direction to get from
$x^{(k-1)}$ to $x^{(k)}$.  Of $x^{(k)}$ is better than any surrounding
point, we decrease $\Delta$ and try again.  More generally, we would
evaluate $\phi(x^{(k)} + d)$ for $d \in \mathcal{G}(\Delta)$, a
{\em generating set} of directions with some scale factor $\Delta$.

\section{Summarizing thoughts}

Direct search methods have been with us for more than half a century:
the original Hook-Jeeves paper was from 1961, and the Nelder-Mead
paper goes back to 1965.  These methods are attractive in that they
require only the ability to compute objective function values, and can
be used with ``black box'' codes -- or even with evaluations based on
running a physical experiment!  Computing derivatives requires some
effort, even when automatic differentiation and related tools are
available, and so gradient-free approaches may also be attractive
because of ease-of-use.

Gradient-free methods often work well in practice for solving
optimization problems with modest accuracy requirements.  This is true
even of methods like Nelder-Mead, for which there are examples of very
nice functions (smooth and convex) for which the method is guaranteed
to mis-converge.  But though the theoretical foundations for these
methods have gradually improved with time, the theory for
gradient-free methods is much less clear-cut than the theory for
gradient-based methods.  Gradient-based methods also have a clear
advantage at higher accuracy requirements.

Gradient-free methods do {\em not} free a user from the burden of
finding a good initial guess.  Methods like Nelder-Mead and pattern
search will, at best, converge to local minima.  Methods such as
simulated annealing may have better luck in finding global minima,
but it is still a hard problem in general.  Gradient-free methods may
also have difficulty with functions that are discontinuous, or that
have large Lipschitz constants.

In many areas in numerics, an ounce of analysis pays for a pound of
computation.  If the computation is to be done repeatedly, or must be
done to high accuracy, then it is worthwhile to craft an approach that
takes advantage of specific problem structure.  On the other hand,
sometimes one just wants to do a cheap exploratory computation to
get started, and the effort of using a specialized approach may not
be warranted.  An overview of the options that are available is
useful for approaching these tradeoffs intelligently.

\section{References}

Our textbook does not have much discussion of gradient-free
optimization.  For further reading at the same level as these notes
(though by a much more knowledgable authority), I recommend ``A view
of algorithms for optimization without derivatives'' by
M.~J.~D.~Powell (2007).  There is also a beautiful survey of direct
search methods by Kolda, Lewis, and Torczon from 2003 (``Optimization
by direct search: new perspectives no some classical and modern
methods,'' {\em SIAM Review}, vol 45, pp.~385--482).
For more detail, try {\em Introduction to
  Derivative-Free Optimization} by Conn, Scheinberg, and Vicente
(SIAM, 2009).  The Cornell library subscribes to SIAM's eBook service,
so if you are on campus, you can get to the electronic version of this
book.

\end{document}
