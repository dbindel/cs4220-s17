\documentclass[12pt, leqno]{article} %% use to set typesize
\usepackage[utf8]{inputenc}
\usepackage[russian]{babel}

\usepackage{amsfonts}
\usepackage{amsmath}
\usepackage{amssymb}
\usepackage{fancyhdr}
\usepackage{hyperref}
\usepackage{tikz}
\usepackage{pgfplots}
\usepackage{listings}

\newcommand{\bbR}{\mathbb{R}}
\newcommand{\bbC}{\mathbb{C}}
\newcommand{\calV}{\mathcal{V}}
\newcommand{\calW}{\mathcal{W}}
\newcommand{\ddiag}{\operatorname{diag}}
\newcommand{\fl}{\operatorname{fl}}
\newcommand{\macheps}{\epsilon_{\mathrm{mach}}}
\newcommand{\matlab}{\textsc{Matlab}}

\newcommand{\hdr}[2]{
  \pagestyle{fancy}
  \lhead{Bindel, Spring 2016}
  \rhead{Numerical Analysis (CS 4220)}
  \fancyfoot{}
  \begin{center}
    {\large{\bf Notes for #1}}
  \end{center}
  \lstset{language=matlab,columns=flexible}  
}


\begin{document}
\hdr{2016-04-27}

\section*{Seeking structure}

For the past three weeks, we have discussed rather general-purpose
optimization methods for nonlinear equation solving and optimization.
In practice, of course, we should look at our problems to see if they
have structure we can use in specialized algorithms that are faster or
more robust than the general-purpose methods.  At one end of the
spectrum, there are specialized solvers for specific problem types,
such as linear programming problems, linear least-squares problems
with $\ell^1$ regularizers (LASSO), or non-negative least squares
problems.  More generally, there are less-specialized solvers (and
proof techniques) for more general classes such as convex optimization
problems.  This week, we focus on two classes of structured nonlinear
equation solvers and optimization methods with deep connections to
linear algebraic ideas from the start of the semester.

\section*{Eigenvalue problems and optimization}

There is a sharp division in the optimization world between
{\em convex} problems and {\em nonconvex} problems.  The former are
tractable; the latter are often very hard indeed.  The symmetric
eigenvalue problem is in a sweet spot between the two: it covers
a broad class of interesting non-convex problems, but we know how
to solve it efficiently.

{\em NB}: I talked about Rayleigh quotients and about graph
partitioning in class, but not the other examples.

\subsection*{The marvelous Rayleigh quotient}

In our discussion of eigenvalue problems, we have already seen
the {\em Rayleigh quotient}
\[
  \rho_A(x) = \frac{x^T A x}{x^T x}.
\]
We commented earlier in the semester that the stationary points
of the Rayleigh quotient are the eigenvectors of $A$, and the
stationary values are the corresponding eigenvalues.

Because $\rho_A(\alpha x) = \rho_A(x)$ for any nonzero $\alpha$,
we can rephrase the problem of finding stationary points of
the Rayleigh quotient as finding critical points of $x^T A x$
constrained to the unit sphere.  Using Lagrange multipliers,
this is equivalent to finding stationary points (with respect
to $x$ and $\lambda$) of
\[
  L(x,\lambda) = x^T A x - \lambda (x^T x - 1).
\]
The gradient of $L$ with respect to $x$ is
\[
  \nabla_x L = 2(Ax-\lambda x);
\]
and so we see that the eigenvalue can also be interpreted as a
Lagrange multiplier.

More generally, we can consider minimizing $x^T A x$ subject
to the constraint that $x^T M x = 1$; the constrained fixed point
equations then take the form of a {\em generalized eigenvalue problem}:
\[
  Ax = \lambda M x.
\]
If $M$ is symmetric and positive definite, we can convert this
generalized problem to a standard eigenvalue problem by a change of
variables.  However, it is often useful to keep such problems in
generalized form, as the conversion to a standard eigenvalue problem
may destroy sparsity.

\subsection*{Quadratic constraints}

What if we move a step beyond the pure quadratic problem?
For example, consider
\[
  \mbox{minimize } \frac{1}{2} x^T A x - b^T x \mbox{ s.t.~} x^T x = 1.
\]
The KKT conditions are
\begin{align*}
  (A-\lambda I) x &= b \\
  x^T x &= 1.
\end{align*}
This is not itself an eigenvalue problem, but we can get there.
Note that
\[
  b^T (A-\lambda I)^{-2} b-1 = 0,
\]
which, with a little cleverness, we can re-interpret as a zero
in the final Schur complement of
\[
  \begin{bmatrix}
  (A-\lambda)^2 & b \\
  b^T & 1
  \end{bmatrix}.
\]
Eliminating the last variable instead of the first gives the
{\em quadratic eigenvalue problem}
\[
  \left[ (A-\lambda I)^2 - b b^T \right] u = 0,
\]
where $v = (A-\lambda u)$ is proportional to a constrained stationary
point.  We can also go one step further and reduce to a standard
eigenvalue problem:
\[
\begin{bmatrix}
  (A-\lambda I) & -I \\
  -bb^T & (A-\lambda I)
\end{bmatrix}
\begin{bmatrix}
  u \\ v
\end{bmatrix} = 0.
\]
This gives us the following characterization of all stationary points
of the constrained problem: for each real solution to the eigenvalue
problem
\[
\begin{bmatrix}
  A & -I \\
  -bb^T & A
\end{bmatrix}
\begin{bmatrix}
  u \\ v
\end{bmatrix} =
\lambda
\begin{bmatrix}
  u \\ v
\end{bmatrix},
\]
the vectors $\pm v/\|v\|$ are stationary points for the constrained
problem.  In this case, it is faster to deal directly with
the quadratic problem; in fact, it is reasonably straightforward to
use a divide-and-conquer idea to solve this problem in $O(n^2)$ time
after a symmetric eigendecomposition of $A$.  But this is perhaps
a topic for a different time.

What if we went one step more complicated?  For example, we might
try to optimize subject to a constraint $x^T M x - 2 d^T x + c = 0$.
But if $M$ is symmetric and positive definite with the Cholesky
factorization $M = R^T R$, then the change of variables
\[
  x = R^{-1} (R^{-T} d + z)
\]
gives us the constraint equation
\[
  z^T z = -\left( \|R^{-T} d\|^2 + c \right)
\]
and a quadratic objective function.  Hence, we can still deal with
this case by reduction to an eigenvalue problem.
On the other hand, if we have more than one quadratic constraint,
things become rather more complicated.

\subsection*{Optimal matrix approximations}

A wide variety of {\em matrix nearness} problems can be converted to
eigenvalue problems.  Often this is an optimization with a quadratic
objective and quadratic constraints in disguise.  As a few examples,
we mention the nearest low-rank matrix to a given target (computed by
the SVD via the Eckart-Young theorem), the nearest positive
semi-definite matrix to a given symmetric matrix (typically computed
via a symmetric eigenvalue decomposition), the nearest orthogonal
matrix to a given matrix (the solution to an {\em orthogonal
  Procrustes problem}), and the distance between a matrix and the
nearest unstable matrix (computed via the Byers\footnote{%
  Ralph Byers was a student of Charlie Van Loan who went on to do
  really interesting work in numerical linear algebra, and
  particularly on eigenvalue problems.  I met him while I was a
  graduate student at Berkeley; he was busy putting his aggressive
  early deflation ideas into the LAPACK nonsymmetric eigensolver.  He
  was a kind and interesting gentleman.  He passed away in December
  2007, just less than a month after Gene Golub's death. He is still
  missed.  }-Boyd-Balikrishnan algorithm).  Nick Higham has written
a good deal on matrix nearness problems; see, e.g.~the survey article
``\href{http://www.maths.manchester.ac.uk/~higham/narep/narep161.pdf}{Matrix Nearness Problems and Applications}.''

\subsection*{Dimensionality reduction}

Students often first encounter the SVD when learning about principal
components analysis (PCA), one of the basic forms of dimensionality
reduction.  There are a variety of other dimensionality reduction
methods that, like PCA, turn into {\em trace optimization} problems:
\[
  \mbox{optimize } \operatorname{tr}(X^T A X) \mbox{ s.t.~} X^T X = I,
\]
where the optimization may be a minimization or maximization depending
on the context (and the matrix $A$).  We will spell out this example
in a little more detail as a reminder of the value of variational
notation.  We can write the augmented Lagrantian in this case as
\[
  L(X,M) = 
  \operatorname{tr}(X^T A X) - \operatorname{tr}(M^T (X^T X-I))
\]
where $M$ is now a {\em matrix} of Lagrange multipliers, one for each
of the scalar constraints in the matrix equation $X^TX = I$.  Taking
variations gives
\[
\delta L =
2 \operatorname{tr}(\delta X^T (AX-XM)) +
\operatorname{tr} (\delta M^T (X^TX-I)),
\]
which gives us the equation
\[
  AX = XM.
\]
for some $M$.  This is an {\em invariant subspace} equation.  The
equations allow $X$ is an arbitrary orthonormal basis for the subspace
in question (the subspace associated with the largest or smallest
eigenvalues depending whether we are interested in trace minimization
or maximization).

For a good overview of the connections between trace optimization and
dimensionality reduction, we refer to
``\href{http://www-users.cs.umn.edu/~saad/PDF/umsi-2009-31.pdf}{Trace
  optimization and eigenproblems in dimension reduction methods}'' by
Kokiopoulou, Chen, and Saad.

\subsection*{Combinatorial connections}

We have already seen that we can optimize a quadratic subject to one
quadratic constraint in the real case.  Integer or binary quadratic
programs are harder, but we can often use a {\em spectral relaxation}
to approximately solve these programs.  As an example, we describe
a standard spectral method for a classic problem in graph theory:
bisecting a graph with as few cut edges as possible.

Consider an undirected graph $G = (V,E)$, $E \subset V \times V$.
We partition $V$ into disjoint subsets $V^+$ and $V^-$ by a
mapping $u : V \rightarrow \{ \pm 1 \}$.  The number of edges
going between $V^+$ and $V^-$ is
\[
\frac{1}{4} \sum_{(i,j) \in E} (u_i-u_j)^2 =
\frac{1}{2} u^T L u
\]
where $L$ is the {\em graph Laplacian matrix}
\[
L_{ij} = \begin{cases}
  \mbox{degree of } i, & i = j \\
  -1, & i \neq j \mbox{ and } (i,j) \in E, \\
  0, & \mbox{otherwise}.
\end{cases}
\]
The difference in the size of $V^+$ and $V^-$ is $e^T u = \sum_i u_i$.
Hence, bisection with a minimum cut corresponds to the binary
quadratic program
\[
  \mbox{minimize } u^T L u \mbox{ s.t.~} e^T u = 0 \mbox{ and } u \in
  \{ \pm 1 \}^n.
\]
The general graph partitioning problem is NP-hard; but the problem
is not in the objective or the constraint that $e^T u = 0$, but the
constraint that $u \in \{ \pm 1 \}^n$.  We can {\em relax} this
constraint to the condition that $\sum_i u_i^2 = n$; that is,
\[
\mbox{minimize } u^T L u \mbox{ s.t.~} e^T u = 0 \mbox{ and }
  \|u\|^2 = n, u \in \bbR^n.
\]
The vector $e$ is itself a null vector of $L$; if the graph is
connected, all other eigenvalues of $L$ are positive.  The solution to
the relaxed problem is that $u$ is a scaled eigenvector corresponding
to the second-smallest eigenvalue $\lambda_2(L)$; the value of $u^T L
u$ is $n \lambda_2(L)$.  Because the continuous optimization is over a
larger set than the discrete optimization, $n \lambda_2(L)$ is a lower
bound on the minimum number of edges needed for bisection.  The
eigenvalue $\lambda_2(L)$ is sometimes called the {\em algebraic
  connectivity} of the graph; when it is large, there are no small
cuts that can bisect the graph  The corresponding eigenvector is
the {\em Fiedler vector}.  Although the entries of the Fiedler vector
are no longer $\pm 1$, using the sign of the entries as a method of
partitioning often works quite well.  This is the basis
of {\em spectral partitioning}.

\section*{Eigenvalues and global root finding}

So far, we have discussed spectral approaches to optimization
problems.  But spectral methods are also relevant to nonlinear
equation solving, particularly in one variable.  The basic picture is:
\begin{itemize}
\item
  If $f : [a,b] \rightarrow \bbR$ is a smooth function, we
  approximate it by a polynomial $p$ to arbitrary accuracy.
\item
  We re-interpret the problem of finding roots of $p$ as the problem
  of finding eigenvalues of $A$ s.t.~$p(z) \propto \det(z I-A)$.
\end{itemize}
There is a standard trick to finding a matrix eigenvalue problem
corresponding to a polynomial root-finding problem; it is the
same trick that's used to convert a differential equation to
first-order form.  Define a recurrence relation for powers
\[
  v_j = \lambda^j = \lambda v_{j-1}
\]
Then we can rewrite
\[
  0 = p(\lambda) = \lambda^d + \sum_{j=0}^{d-1} a_j \lambda^j
\]
as
\[
  0 = \lambda v_{d-1} + \sum_{j=0}^{d-1} a_j v_j.
\]
Putting this together with the recurrence relation, we have
\[
\begin{bmatrix}
  -a_{d-1} & -a_{d-2} & \ldots & -a_{1} & -a_{0} \\
  1 \\
    & 1 \\
    &   & \ddots \\
    &   &   & 1 & 0
\end{bmatrix}
\begin{bmatrix}
  v_{d-1} \\ v_{d-2} \\ v_{d-3} \\ \vdots \\ v_0
\end{bmatrix} =
\lambda
\begin{bmatrix}
  v_{d-1} \\ v_{d-2} \\ v_{d-3} \\ \vdots \\ v_0
\end{bmatrix}.
\]
This matrix with polynomial coefficients across the top is
a {\em companion matrix}.  It is a highly structured Hessenberg
matrix, although exploiting the structure in a numerically stable
way turns out to be nontrivial.  The MATLAB {\tt roots} command
computes roots of a polynomial by running the ordinary QR eigensolver
on such a matrix.

More generally, if $p$ is expressed in terms of a basis that can be
evaluated by some recurrence relation (e.g.~Chebyshev or Legendre
polynomials), then there is an associated Hessenberg {\em confederate
  matrix} whose eigenvalues are the roots of the polynomial.  For the
Chebyshev polynomials, the confederate matrix is sometimes called a
{\em comrade matrix}\footnote{Товарищ?}; there is also something
called a {\em colleague matrix}.  The trick is also not restricted to
expansions in terms of polynomials; for example, the problem of
finding roots of a function that is well approximated by a finite
Fourier series can similarly be converted into an eigenvalue
problem\footnote{%
  See, for example, {\em Solving Transcendental Equations} by
  J.P.~Boyd.  This is a book worth remembering even if you have
  decided you are only mildly interested in equation solving.
  Boyd is an unusually entertaining writer.
}.

\subsection*{Nonlinear eigenvalue problems}

A {\em nonlinear eigenvalue problem} is a problem of the form
\[
  \mbox{Find } (\lambda, u) \mbox{ with } u \neq 0 \mbox{ s.t.~}
  T(\lambda) u = 0.
\]
Your third project is an example of a nonlinear eigenvalue problem
(the eigenvalue is the equilibrium resource level $r$), but there are
many other examples that come from analyzing the stability of linear
differential equations and difference equations.  We can approximate
many nonlinear eigenvalue problems by {\em polynomial} eigenvalue
problems $P(\lambda) u = 0$ where
\[
  P(z) \approx \sum_{j=0}^d z^j A_j.
\]
This problem, too, can be encoded as a matrix eigenvalue problem;
if $A_d$ is nonsingular, we have
\[
\begin{bmatrix}
  -\bar{A}_{d-1} & -\bar{A}_{d-2} & \ldots & -\bar{A}_{1} & -\bar{A}_{0} \\
  I \\
    & I \\
    &   & \ddots \\
    &   &   & I & 0
\end{bmatrix}
\begin{bmatrix}
  v_{d-1} \\ v_{d-2} \\ v_{d-3} \\ \vdots \\ v_0
\end{bmatrix} =
\lambda
\begin{bmatrix}
  v_{d-1} \\ v_{d-2} \\ v_{d-3} \\ \vdots \\ v_0
\end{bmatrix},
\]
where $\bar{A}_j = A_d^{-1} A_j$.

\section*{Spectral summary}

A wide variety of problems can be reduced to or approximated by either
a quadratic program with one quadratic equality constraint or by
a polynomial root-finding problem.  When you see a problem with this
structure, most likely there is an eigenvalue problem lurking nearby.

\end{document}
