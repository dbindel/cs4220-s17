\documentclass[12pt, leqno]{article} %% use to set typesize 
\usepackage{amsfonts}
\usepackage{amsmath}
\usepackage{amssymb}
\usepackage{fancyhdr}
\usepackage{hyperref}
\usepackage{tikz}
\usepackage{pgfplots}
\usepackage{listings}

\newcommand{\bbR}{\mathbb{R}}
\newcommand{\bbC}{\mathbb{C}}
\newcommand{\calV}{\mathcal{V}}
\newcommand{\calW}{\mathcal{W}}
\newcommand{\ddiag}{\operatorname{diag}}
\newcommand{\fl}{\operatorname{fl}}
\newcommand{\macheps}{\epsilon_{\mathrm{mach}}}
\newcommand{\matlab}{\textsc{Matlab}}

\newcommand{\hdr}[2]{
  \pagestyle{fancy}
  \lhead{Bindel, Spring 2016}
  \rhead{Numerical Analysis (CS 4220)}
  \fancyfoot{}
  \begin{center}
    {\large{\bf Notes for #1}}
  \end{center}
  \lstset{language=matlab,columns=flexible}  
}


\begin{document}
\hdr{2017-04-14}

\section{Taylor revisited}

Though we have mentioned optimization problems intermittently through
the last few lectures, we have focused mainly on the problem of
finding the solutions of nonlinear equations.  We now turn to the
problem of finding (local) optima of functions with at least two
continuous derivatives.

Recall the basic Taylor expansion that we outlined before the break;
if $\phi : \bbR^n \rightarrow \bbR$ is $\mathcal{C}^2$ (i.e.~if it is
at least twice continuously differentiable) then we have the expansion
\[
  \phi(x+u) = \phi(x) + \phi'(x) u + \frac{1}{2} u^T H_{\phi}(x) u + o(\|u\|^2)
\]
where $\phi'(x) \in \bbR^{1 \times n}$ is the derivative of $\phi$ and
$H_{\phi}$ is the {\em Hessian matrix} consisting of second
derivatives:
\[
  (H_\phi)_{ij} = \frac{\partial \phi}{\partial x_i \partial x_j}.
\]
The {\em gradient} $\nabla \phi(x) = \phi'(x)^T$ is a column vector
(rather than a row vector).

If $\nabla \phi(x) \neq 0$ then $\nabla \phi(x)$ and $-\nabla \phi(x)$
are the directions of steepest ascent and descent, respectively.
If $\nabla \phi(x) = 0$, then we say $x$ is a {\em stationary point}
or {\em critical point}.  The first derivative test says that if $x$
minimizes $\phi$ (and $\phi$ is differentiable) then the gradient of
$x$ must be zero; otherwise, there is a ``downhill'' direction, and
a point near $x$ achieves a smaller function value.

A stationary point does not need to be a local minimizer; it might
also be a maximizer, or a saddle point.  The {\em second} derivative
test says that for a critical point $x$ to be a (local) minimizer, the
Hessian $H_{\phi}(x)$ must be at least positive semi-definite at a
(local) minimizer.  If $x$ is a stationary point and $H_{\phi}$ is
strictly positive definite, then $x$ must be a local minimizer; in
this case, we call $x$ a {\em strong} local minimizer.

One approach to the problem of minimizing $\phi$ is to run Newton
iteration on the critical point equation $\nabla \phi(x) = 0$.
The Jacobian of the function $\nabla \phi(x)$ is simply the Hessian
matrix, so Newton's iteration for finding the critical point is just
\[
  x_{k+1} = x_k - H_{\phi}(x_k)^{-1} \nabla \phi(x_k).
\]
We can derive this in the same way that we derived Newton's iteration
for other nonlinear equations; or we can derive it from finding the
critical point of a quadratic approximation to $\phi$:
\[
  \hat{\phi}(x_k+p_k) =
    \phi(x_k) + \phi'(x_k) p_k + \frac{1}{2} p_k^T H_{\phi}(x_k) p_k.
\]
The critical point occurs for $p_k = -H_{\phi}(x_k)^{-2} \nabla \phi(x_k)$;
but this critical point is a strong local minimum iff $H_{\phi}(x_k)$
is positive definite.

There are a few reasons we might want to dig deeper:
\begin{itemize}
\item As with other systems of nonlinear equations, we might prefer
  to avoid a Newton iteration because of the cost of factoring the
  Jacobian (in this case, the Hessian matrix, which is the Jacobian
  of $\nabla \phi$).
\item We can take advantage of the fact that this is {\em not} a
  general system of nonlinear equations in devising and analyzing
  methods.
\item If we only seek to solve the critical point equation, we might
  end up finding a maximizer or saddle point as easily as a minimizer.
\end{itemize}
For this reason, we will discuss a different class of iterations, the
(scaled) gradient descent methods and their relatives.  At the end of
the day, we will see many of the same ideas that we saw when treating
nonlinear equations, but we will get to them by a slightly different
path.

\section{Gradient descent}

One of the simplest optimization methods is the {\em steepest descent}
or {\em gradient descent} method
\[
  x_{k+1} = x_k + \alpha_k p_k
\]
where $\alpha_k$ is a {\em step size} and $p_k = -\nabla \phi(x_k)$.
To understand the convergence of this method, consider gradient
descent with a fixed step size $\alpha$ for the quadratic model problem
\[
  \phi(x) = \frac{1}{2} x^T A x + b^T x + c
\]
where $A$ is symmetric positive definite.  
We have computed the gradient for a quadratic before:
\[
  \nabla \phi(x) = Ax + b,
\]
which gives us the iteration equation
\[
  x_{k+1} = x_k - \alpha (A x_k + b).
\]
Subtracting the fixed point equation
\[
  x_* = x_* - \alpha (A x_* + b)
\]
yields the error iteration
\[
  e_{k+1} = (I-\alpha A) e_k.
\]
If $\{ \lambda_j \}$ are the eigenvalues of $A$, then the
eigenvalues of $I-\alpha A$ are $\{ 1-\alpha \lambda_j \}$.
The spectral radius of the iteration matrix is thus
\[
  \min \{ |1-\alpha \lambda_j| \}_j =
  \min \left( |1-\alpha \lambda_{\min}|, |1-\alpha \lambda_{\max}| \right).
\]
The iteration converges provided $\alpha < 1/\lambda_{\max}$, and the
optimal $\alpha$ is
\[
  \alpha_* = \frac{2}{\lambda_{\min} + \lambda_{\max}},
\]
which leads to the spectral radius
\[
  1 - \frac{2 \lambda_{\min}}{\lambda_{\min} + \lambda_{\max}} =
  1 - \frac{2}{1 + \kappa(A)}
\]
where $\kappa(A) = \lambda_{\max}/\lambda_{\min}$ is the condition
number for the (symmetric positive definite) matrix $A$.  If $A$
is ill-conditioned, then, we are forced to take very small steps
to guarantee convergence, and convergence may be
heart breakingly slow.  We will get to the minimum in the long run
--- but, then again, in the long run we all die.

The behavior of steepest descent iteration on a quadratic model
problem is indicative of the behavior more generally: if $x_*$ is a
strong local minimizer of some general nonlinear $\phi$, then gradient
descent with sufficiently small step size will converge locally to
$x_*$.  But if $H_{\phi}(x_*)$ is ill-conditioned, then one has to
take small steps, and the rate of convergence can be quite slow.

Not all problems are terrible ill-conditioned, and so in many cases
simple gradient descent algorithms can work quite well.  For
ill-conditioned problems, though, we would like to change something
about the algorithm.  One approach is to keep the gradient descent
direction and adapt the step size in a clever way; the
Barzelei-Borwein (BB) method and related approaches follow this
approach.  These remarkable methods deserve to be better known,
but in the interest of fitting the course into the semester, we will
turn instead to the problem of choosing better directions.

\section{Scaled gradient descent}

The {\em scaled} gradient descent iteration takes the form
\[
  x_{k+1} = x_k + \alpha_k p_k, \quad M_k p_k = -\nabla \phi(x_k).
\]
where $\alpha_k$ and $p_k$ are the step size and direction, as before,
and $M_k$ is a symmetric positive definite {\em scaling matrix}.
Positive definiteness of $M_k$ guarantees that $p_k$ is a
{\em descent direction}, i.e.~
\[
\phi'(x_k) p_k = \nabla \phi(x_k)^T p_k
= -\nabla \phi(x_k)^T M_k^{-1} \nabla \phi(x_k) < 0;
\]
this in turn guarantees that if $\alpha_k$ is sufficiently small,
$\phi(x_{k+1})$ will be less than $\phi(x_k)$ --- unless $\phi(x_k)$ is
a stationary point (i.e.~$\nabla \phi(x_k) = 0$).

How does scaling improve on simple gradient descent?  Consider again
the quadratic model problem
\[
  \nabla \phi(x) = Ax + b,
\]
and let $M$ and $\alpha$ be fixed.  With a little work, we derive the
error iteration
\[
  e_{k+1} = (I-\alpha MA) e_k
\]
If $\alpha M = A^{-1}$, the iteration converges in a single step!
Going beyond the quadratic model problem, if $H_{\phi}(x_k)$ is
positive definite, we might choose $M_k = H_{\phi}(x_k)$ ---
which would correspond to a Newton step.

Of course, $H_{\phi}(x_k)$ does not have to be positive definite
everywhere!  Thus, most minimization codes based on Newton scaling use
$M_k = H_{\phi}(x_k)$ when it is positive definite, and otherwise use
some modification.  One possible modification is to choose a diagonal
shift $M_k = H_{\phi}(x_k) + \beta I$ where $\beta$ is sufficiently
large to guarantee positive definiteness.  Another common approach is
to compute a {\em modified Cholesky} factorization of $H_{\phi}(x_k)$.
The modified Cholesky algorithm looks like ordinary Cholesky, and is
identical to ordinary Cholesky when $H_{\phi}(x_k)$ is positive
definite.  But rather than stopping when it encounters a negative
diagonal in a Schur complement, the modified Cholesky approach
replaces that element with something else and proceeds.

\section{Modified and quasi-Newton optimizers}

In the last two lectures, we described a variety of Newton-like
methods for solving nonlinear equations that might involve fewer
derivative computations and lower cost less per step than Newton
iteration.  Most of these ideas carry over to optimization problems as
well, but with some twists to ensure that steps always move in a
descent direction.  Some of the variants are:
\begin{itemize}
\item
  {\bf Scaling with an approximate Hessian}: Here we choose $M_k$ to be a
  symmetric positive definite matrix that in some sense approximates
  $H_{\phi}(x_k)$ (at least when the Hessian is positive definite) and
  for which it is easy to solve linear systems.
\item
  {\bf Inexact Newton steps}: Here we compute a (modified) Newton step,
  but inexactly (e.g. using a Krylov subspace method like PCG).
\item
  {\bf Quasi-Newton steps}: The most popular quasi-Newton method for
  optimization is the BFGS method (Broyden-Fletcher-Goldfarb-Shanno).
  The Broyden in BFGS is the same as the Broyden we saw in the popular
  Broyden quasi-Newton method for nonlinear equation solving, but the
  update formula itself is a little different.  We still seek to
  satisfy a secant condition, but in BFGS we also seek to retain
  symmetry and positive definiteness of the approximate Hessian
  matrix.  This is done with a rank two update.

  The limited memory BFGS (L-BFGS) method uses only a fixed set of
  previous iterates in the Hessian approximation, rather than
  considering the entire past convergence history.  L-BFGS is one of
  the most popular methods for large scale optimization.
\end{itemize}

There are also ``Krylov-like'' methods for choosing update directions,
most prominent of which are the nonlinear conjugate gradient methods.
There are many potential nonlinear CG methods; for quadratic objective
functions, they are all equivalent, but they differ when applied to
more general functions.  Newton-like methods may be more effective,
but usually require more memory and computation per step.  Anderson
acceleration (discussed in the last lecture) has also been
successfully applied to optimization methods.

\end{document}
