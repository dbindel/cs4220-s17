\documentclass[12pt, leqno]{article} %% use to set typesize 
\usepackage{amsfonts}
\usepackage{amsmath}
\usepackage{amssymb}
\usepackage{fancyhdr}
\usepackage{hyperref}
\usepackage{tikz}
\usepackage{pgfplots}
\usepackage{listings}

\newcommand{\bbR}{\mathbb{R}}
\newcommand{\bbC}{\mathbb{C}}
\newcommand{\calV}{\mathcal{V}}
\newcommand{\calW}{\mathcal{W}}
\newcommand{\ddiag}{\operatorname{diag}}
\newcommand{\fl}{\operatorname{fl}}
\newcommand{\macheps}{\epsilon_{\mathrm{mach}}}
\newcommand{\matlab}{\textsc{Matlab}}

\newcommand{\hdr}[2]{
  \pagestyle{fancy}
  \lhead{Bindel, Spring 2016}
  \rhead{Numerical Analysis (CS 4220)}
  \fancyfoot{}
  \begin{center}
    {\large{\bf Notes for #1}}
  \end{center}
  \lstset{language=matlab,columns=flexible}  
}


\begin{document}
\hdr{2017-04-17}

\section{Broadening the Basin}

All the methods we have so far discussed for solving nonlinear
equations or optimization problems have the form
\[
  x_{k+1} = x_k + \alpha_k p_k
\]
where $\alpha_k$ is a {\em step size} and $p_k$ is a {\em search
  direction}.  We have described a wide variety of methods for
choosing the search directions $p_k$.  We have also analyzed several
of these methods (or at least pointed to their analysis) under the
assumption that the step sizes were chosen to be $\alpha_k = 1$ (or,
in our analysis of gradient descent, $\alpha_k = \alpha$ some
constant).  But so far, our analyses have all come with the caveat
that convergence is only assured for initial guesses that are ``good
enough.''  We call the set of initial guesses for which a nonlinear
solver or optimizer converges to a given solution $x_*$ the {\em basin
  of convergence} for $x_*$.  In a previous lecture, we have already
discussed some features that make the basin of convergence large or
small for Newton and modified Newton iterations.  Today we begin our
discussion of {\em globalization} methods that allow us to guarantee
convergence even if we lack a good enough initial guess to make our
unguarded iterations converge.

In our discussion today, it will be convenient to focus on
globalization by {\em line search methods} that make intelligent,
adaptive choices of the step size.  Informally, these methods work with any
``reasonable'' method for choosing search directions $p_k$ (which
should at least be descent directions).  An {\em exact line search}
method seeks to minimize $g(\alpha) = \phi(x_k + \alpha p_k)$ by a
one-dimensional optimization; but it turns out that the work required
for exact line search usually does not justify the benefit.  Instead,
we consider {\em inexact line search} methods that choose step sizes
$\alpha_k$ such that the methods:
\begin{itemize}
\item Make significant progress in the downhill direction
  ($\alpha_k$ not too small).
\item But don't step so far they go back uphill
  ($\alpha_k$ not too big).
\end{itemize}
We need to tighten and formalize these conditions a little bit in
order to obtain formal convergence results, but this is the right
intuition.

\section{A series of unfortunate examples}

In order to illustrate the conditions we will require -- and the
limits of our approach -- we will first consider three illustrative
examples.

\subsection{The long march to infinity}

Consider the one-dimensional objective function
\[
  \phi(x) = x \tan^{-1}(x) - \frac{1}{2} \log(1+x^2).
\]
The first and second derivatives of $\phi$ are
\begin{align*}
  \phi'(x) &= \tan^{-1}(x) \\
  \phi''(x) &= \frac{1}{1+x^2}.
\end{align*}
This is a convex function with a unique global minimum $\phi(0) = 0$.
To find this minimum, we might first consider Newton's iteration:
\[
  x_{k+1} = x_k - \frac{\phi'(x)}{\phi''(x)} = x_k - (1+x_k^2) \tan^{-1}(x_k).
\]
The Newton step is always in a descent direction, and
the iteration converges for $|x_0| \leq \xi \approx 1.3917$; here
$\xi$ is the solution to the ``anti-fixed-point'' equation
\[
  -\xi = \xi - (1+\xi^2) \tan^{-1}(\xi).
\]
For any $|x_0| > \xi$, the iterates blow up, alternating between
positive and negative numbers of increasingly wild magnitudes.
The Newton step always goes in the right direction, but it goes too
far.

A simple fix, which works in this case, is to check for progress
and cut the step in half if it is not obtained; that is, we take
\[
  x_{k+1} = x_k - \alpha_k \frac{\phi'(x_k)}{\phi''(x_k)}
\]
where $\alpha_k$ is the first value $2^{-j}$ for $j = 0, 1, \ldots$
that guarantees $\phi(x_{k+1}) < \phi(x_k)$.  The corresponding code
is shown in Figure~\ref{fig:newton-naive-ls}.

\begin{figure}
  \lstinputlisting{code/newton_simple_guard.m}
  \caption{1D Newton optimizer with a naive backtracking line search.}
  \label{fig:newton-naive-ls}
\end{figure}

\subsection{Obscure oscillation}

\begin{figure}
\begin{tikzpicture}
  \begin{axis}[width=\textwidth,height=5cm,xlabel={$k$},ylabel={$x_k$}]
    \addplot table [x=k,y=u] {data/newton_osc.dat};
  \end{axis}
\end{tikzpicture}
\begin{tikzpicture}
  \begin{semilogyaxis}[width=\textwidth,height=5cm,xlabel={$k$},ylabel={$\phi(x_k)-\phi(1)$}]
    \addplot table [x=k,y=d] {data/newton_osc.dat};
  \end{semilogyaxis}
\end{tikzpicture}
\caption{
  Oscillation of Newton for
  $\phi(x) = 19x^2 - 4x^4 + \frac{7}{9} x^6$.
  The iterates jump back and forth between just greater than 1
  and just less than -1 (top), and the objective values are
  monotonically decreasing toward $\phi(1) \approx 15.7778$ (bottom).}
\label{fig:newton-cvg}
\end{figure}

As a second example, consider minimizing the polynomial
\[
  \phi(x) = 19x^2 - 4x^4 + \frac{7}{9} x^6.
\]
The relevant derivatives are
\begin{align*}
  \phi'(x) &= 38x - 16x^3 + \frac{14}{3} x^5 \\
  \phi''(x) &= 38 - 48x^2 + \frac{70}{3} x^4.
\end{align*}
The function is convex --- the minimum value of $\phi''(x)$ is about
$13.3$ --- and there is a unique global minimum at zero.  So what
happens if we start Newton's iteration at $x_0 = 1.01$?

The progress of the iteration is shown in Figure~\ref{fig:newton-cvg}.
If we look only at the objective values, we seem to be making
progress; each successive iterates is smaller than the preceding one.
But the values of $\phi$ are not converging toward zero, but
toward $\phi(\pm 1) = 142/9 \approx 15.778$!  The iterates themselves
slosh back and forth, converging to a limit cycle where the iteration
cycles between $1$ and $-1$.  Furthermore, while this polynomial was
carefully chosen, the qualitative cycling behavior is robust to small
perturbations to the starting guess and to the polynomial
coefficients.  Though it appears to be making progress, the iteration
is well and truly stuck.

The moral is that decreasing the function value from
step to step is not sufficient.  Though just insisting on a decrease
in the objective function from step to step will give convergence for
many problems, we need a stronger condition to give any sort of guarantee.
But this, too, can be fixed.

\subsection{The planes of despair}

\begin{figure}
\begin{tikzpicture}
  \begin{axis}[width=\textwidth,height=5cm,xlabel={$x$},ylabel={$\phi(x)$}]
    \addplot[mark=none] table {data/double_pass_fun.txt};
  \end{axis}
\end{tikzpicture}
\caption{Plot of $\phi(x) = \exp(-x^2/2)-\exp(-x^4/4)$}
\label{fig:doublepass}
\end{figure}

As a final example, consider the function
\[
  \phi(x) = \exp(-x^2/2)-\exp(-x^4/4),
\]
plotted in Figure~\ref{fig:doublepass}.
This function has two global minima close at around $\pm 0.88749$
separated by a local maximum at zero, and two global maximum around
$\pm 1.8539$.  But if we always move in a descent direction, then any
iterate that lands outside the interval $[-1.8539, 1.8539]$ dooms the
iteration to never enter that interval, and hence never find either of
the minima.  Instead, most solvers are likely to march off toward
infinity until the function is flat enough that the solver decides it
has converged and terminates.  This is the type of problem that we do
{\em not} solve with globalization, and illustrates why good
initial guesses remain important even with globalization.

\section{Backtracking search and the Armijo rule}

The idea of a backtracking search is to try successively
shorter steps until reaching one that makes ``good enough''
progress.  The step sizes have the form
$\alpha \rho^j$ for $j = 0, 1, 2, \ldots$ where $\alpha$ is the
default step size and $\rho < 1$ is a backtracking factor (often
chosen to be $0.5$).  As we saw in our examples, we need a more
stringent acceptance condition than just a decrease in the function
value --- otherwise, we might get unlucky and end up converging to
a limit cycle.  That stronger condition is known as the
{\em sufficient decrease} or the {\em Armijo rule}.  For optimization,
this condition takes the form
\[
  \phi(x_k + \alpha p_k) \leq \phi(x_k) + c_1 \alpha \phi'(x_k) p_k
\]
for some $c_1 \in (0,1)$.  Assuming that $p_k$ is a descent direction,
this condition can always be satisfied for small enough $\alpha$,
as Taylor expansion gives
\[
  \phi(x_k + \alpha p_k) = \phi(x_k) + \alpha \phi'(x_k) p_k + o(\alpha).
\]

In practice, it is fine to choose $c_1$ to
be quite small; the value of $10^{-4}$ is suggested by several
authors.  This condition can always be satisfied for small enough
choices of $\alpha$.  Such a line search algorithm looks much the
same as the naive line search that we described earlier, but with
a more complicated termination condition on the line search loop:
\begin{lstlisting}
  % Given a full step p and current value pk, current gradient gk
  a = aref;
  phip = phi(x+a*p);
  slope = gk'*p;

  % Reduce the step until the Armijo condition is satisfied
  while phip > phik + c1*a*slope
    a = rho*a;
    phip = phi(x+a*p);
  end

  % Update x and reference objective
  x = x + a*p;
  phik = phip;
\end{lstlisting}
The contraction factor $\rho$ may be chosen a priori
(e.g.~$\rho = 0.5$), or it may be chosen dynamically from some range
$[\rho_{\min}, \rho_{\max}]$ where $0 < \rho_{\min} < \rho_{\max} < 1$.

\section{The curvature condition}

Backtracking line search is not the only way to choose the step
length.  For example, one can also use methods based on a polynomial
approximation to the objective function along the ray defined by the
search direction, and this may be a better choice for non-Newton.
In this case, we need to guard not only against steps that are too
long, but also steps that are too short.  To do this, it is helpful
to enforce the {\em curvature condition}
\[
  \frac{\partial \phi}{\partial p_k}(x_k+\alpha p_k) \geq c_2
  \frac{\partial \phi}{\partial p_k}(x_k)
\]
for some $0 < c_1 < c_2 < 1$.  The curvature condition simply says
that if the slope in the $p_k$ direction at a proposed new point
is almost the same as the slope at the starting point, then we
should keep going downhill!  Together, the sufficient descent
condition and the curvature conditions are known as the
{\em Wolfe conditions}.  Assuming $\phi$ is at least continuously
differentiable and that it is bounded from below along the ray
$x_k+\alpha p_k$, it is always possible to choose a step size $\alpha$
that satisfies the Wolfe conditions.

\section{Armijo and nonlinear equations}

While the Armijo rule evolved in optimization theory, the same concept
of sufficient decrease of the function applies in nonlinear equation
solving.  To measure progress, we typically monitor the residual norm
$\|f(x)\|$.  If $p_k = -f'(x_k)^{-1} f(x_k)$ is the Newton direction
from a point $x_k$, a linear model of $f$ predicts that
\[
  \|f(x_k + \alpha p_k)\| \approx
  \|f(x_k) + \alpha f'(x_k) p_k\| =
  (1-\alpha) \|f(x_k)\|;
\]
that is, the predicted decrease is by $\alpha \|f(x_k)\|$.
We insist on some fraction of the predicted decrease as a
sufficient decrease to accept a step, yielding the condition
\[
  \|f(x_k + \alpha p_k)\| \leq (1-c_1 \alpha) \|f(x_k)\|.
\]
We don't have to take a Newton step to use this criteria; it
is sufficient that the step satisfy an inexact Newton criterion
such as
\[
  \|f(x_k) + f'(x_k) p_k\| \leq \eta \|f(x_k)\|
\]
for some $\eta < 1$.

\section{Global convergence}

In general, if we seek to minimize an objective $\phi$ that is $C^1$ with
a Lipschitz first derivative and
\begin{itemize}
\item We use one of the line search algorithms sketched above
  (backtracking line search or line search satisfying the Wolfe
  conditions),
\item The steps $p_k$ are {\em gradient related} ($\|p_k\| \geq m
  \|\nabla \phi(x_k)\|$ for all $k$ -- they don't shrink too fast),
\item The angles between $p_k$ and $-\nabla \phi(x_k)$ are acute
  and uniformly bounded away from away from ninety degrees.
\item The iterates are bounded (it is sufficient that the set of
  points less than $\phi(x_0)$ is bounded),
\end{itemize}
then we are guaranteed global convergence to a stationary point.  Of
course, even with all these conditions, we might converge to a saddle
or a local minimizer that is different from the solution we hoped to
find; and we are not guaranteed {\em fast} convergence.  So the choice
of initial guess, and the choice of iterative methods, still matters
a great deal.  Nonetheless, the point remains that an appropriately
chosen line search can help improve the convergence behavior of the
methods we have described so far by quite a bit.

We have not described the full range of possible line searches.
In addition to algorithms that inexactly minimize the objective
with espect to the line search parameter,
there has also been some work on {\em non-monotone} line search
algorithms that allow increases in the function values, as long as
progress is made in some more averaged sense (e.g.~the new point
has a objective function value smaller than the maximum objective
function for the past few points).  This is useful for improving
convergence speed on some hard problems, and is useful in the context
of particular classes of methods such as spectral projected gradient
(about which we will say nothing in this class other than the name).

\end{document}
